\documentclass{article}
\usepackage[utf8]{inputenc}
\usepackage{enumitem}
\usepackage{tocloft} % Optional: Falls du das Inhaltsverzeichnis schöner formatieren willst
\usepackage{ngerman}
\usepackage{graphicx} % Required for inserting images
\usepackage{longtable}
\usepackage{array}
\usepackage{parskip}
\usepackage{float}

\title{\textbf{EventHub}}
\author{\textbf{Felix Franke 890890890}\\ \textbf{Markus Wurms 890890890}}
\date{\textbf{6. Februar 2026}}

\begin{document}

\maketitle

\newpage

\tableofcontents

\newpage

\vspace{\baselineskip}  % Zeilenumbruch der aktuellen Schriftgröße

\newpage

\section{Einleitung}


\subsection{Ziel der Ausarbeitung}

\subsection{Systemüberblick}

\subsection{Aufgabenteilung \textbf{Felix Franke}}

\subsubsection{Fachliche Grundlagen}
\begin{itemize}
    \item Analyse der Aufgabenstellung und fachlichen Anforderungen
    \item Festlegung eines aussagekräftigen Produktnamens
    \item Identifikation und Beschreibung aller Akteure und Rollen
    \item Erstellung des Glossars der Fachdomäne
\end{itemize}

\subsubsection{Funktionale Anforderungen}

\begin{itemize}
    \item Definition funktionaler Gruppen (z. B. Benutzerverwaltung, Gruppenmanagement, Eventplanung)
    \item Ausformulierung aller User Stories je funktionaler Gruppe unter Angabe von Rolle, Ziel, Nutzen, Akzeptanzkriterien und MoSCoW-Priorisierung
    \item Ergänzung fehlender funktionaler Anforderungen (Plausibilitätsprüfung)
\end{itemize}

\subsubsection{Nicht-funktionale Anforderungen}

\begin{itemize}
    \item Beschreibung der Qualitätsmerkmale nach ISO 25010 (Usability, Sicherheit, Performance,
Wartbarkeit etc.)
    \item Rechtliche und organisatorische Anforderungen (DSGVO, Zustimmungspflichten)
\end{itemize}

\subsubsection{GUI \& Usability}

\begin{itemize}
    \item Entwurf der GUI‑Mockups für Webbrowser
    \item Entwurf der GUI‑Mockups für mobile Endgeräte
    \item Zustandsdiagramme für die Navigation (je Akteur/Rolle)
    \item Zuordnung der Screens zu den jeweiligen User Stories
    \item Beschreibung der angewandten UI-/UX‑Designprinzipien
\end{itemize} 

\subsection{Aufgabenteilung \textbf{Markus Wurms}}

\subsubsection{Domänen- \& Datenmodellierung}

\begin{itemize}
    \item Ableitung der Domänenklassen aus den User Stories
    \item Erstellung des fachlichen Domänenmodells (UML‑Klassendiagramm)
    \item Modellierung von Beziehungen, Multiplizitäten und Vererbungen
    \item Ergänzung sinnvoller Attribute und Klassen
\end{itemize}

\subsubsection{Zustände \& Datenpersistenz}

\begin{itemize}
    \item Zustandsdiagramme für zentrale Klassen (z. B. Event, Rechnung, Aufgabe)
    \item Überführung des fachlichen Modells in ein logisches Datenmodell
    \item Definition von Bewegungsdaten (Zahlungen, Buchungen, Statusänderungen)
    \item Erstellung der CRUD‑Matrix (User Stories $\leftrightarrow$ Klassen)
\end{itemize}

\subsubsection{Systemarchitektur \& Schnittstellen}

\begin{itemize}
    \item Beschreibung der Drei‑Schichten‑Architektur (Client – Server – Datenbank)
    \item UML‑Verteilungsdiagramm (Hardware-/Systemlandschaft)
    \item Definition der REST‑API (Ressourcen, Endpunkte, HTTP‑Methoden)
    \item Sequenzdiagramme für ausgewählte Kern-Use‑Cases
\end{itemize}

\subsubsection{Sicherheit \& Datenschutz}

\begin{itemize}
    \item Sicherheitsanalyse mittels STRIDE
    \item Beschreibung der Schutzmaßnahmen (Authentifizierung, Autorisierung, Verschlüsselung)
    \item Technische Umsetzung von DSGVO‑Anforderungen (Zugriffsschutz, Datenminimierung,
Löschkonzepte)
\end{itemize}


\section{Szenarien-Sicht (Use-Case View)}

\subsection{Stakeholder und Benutzerrollen}

\subsection{Überblick über die User Stories}

\subsection{Use-Case-Diagramm}

\subsection{Beschreibung zentraler Use Cases}

\subsubsection{Gruppe erstellen und verwalten}

\subsubsection{Gruppe per Code beitreten}

\subsubsection{Event planen und durchführen}

\subsubsection{Finanzverwaltung (Einzahlungen/Ausgaben)}


\section{Logische Sicht}

\subsection{Ableitung der Domänenklassen aus den User Stories}

Die Domänenklassen des Systems EventHub wurden systematisch aus den fachlichen
Anforderungen und User Stories abgeleitet. Ziel war es, zentrale Konzepte des
Anwendungsbereichs als eigenständige Klassen zu identifizieren und ihre Beziehungen
zueinander fachlich korrekt abzubilden.

Aus den User Stories zur Benutzer- und Gruppenverwaltung (z.\,B. Registrierung,
Gruppenerstellung, Gruppenbeitritt) ergeben sich die Klassen \textit{Benutzer},
\textit{Gruppe}, \textit{Mitgliedschaft} und \textit{Rolle}. Die Mitgliedschaft
modelliert dabei explizit die Beziehung zwischen Benutzern und Gruppen inklusive
Status und Rolle.

User Stories zur Eventplanung und -durchführung führen zur Klasse \textit{Event}
sowie zu abhängigen Konzepten wie \textit{Aufgabe}, \textit{AufgabenZuweisung},
\textit{Medium} und \textit{Kommentar}. Diese Klassen ermöglichen die strukturierte
Organisation und Dokumentation von Events.

Funktionen zur Kommunikation und Abstimmung innerhalb von Gruppen werden durch
Notizen und Umfragen abgebildet. Daraus ergeben sich die Klassen \textit{Notiz}
mit ihren Spezialisierungen (Freitext-, Termin- und Ausgaben-Notiz) sowie
\textit{Umfrage}, \textit{UmfrageOption} und \textit{Stimme}.

Die User Stories zur Finanzverwaltung (Einzahlungen, Rückzahlungen, Kassenübersicht)
resultieren in den Klassen \textit{Einzahlung}, \textit{Rueckzahlung} und
berichtenden Strukturen wie dem \textit{Kassensturz}.

Erweiterte User Stories für Dienstleisterprozesse und Ressourcenplanung führen zu
Domänenklassen wie \textit{Dienstleister}, \textit{Anfrage}, \textit{Angebot},
\textit{Rechnung}, \textit{Mahnung} sowie \textit{Material}, \textit{Mitarbeiter}
und den zugehörigen Buchungs- und Einsatzklassen.

Die so identifizierten Domänenklassen bilden die fachliche Grundlage für das
Klassendiagramm und dienen als Ausgangspunkt für das logische Datenmodell.

\subsection{Fachliches Domänenmodell (UML-Klassendiagramm)}

\begin{figure}[H]
    \centering
    \includegraphics[width=1.3\textwidth]{Klassendiagramm1.png}
    \caption{Klassendiagramm1.png}
    \label{fig:klassendiagramm}
\end{figure}

\subsection{Beziehungen, Multiplizitäten und Vererbungen}

\subsection{Logisches Datenmodell}

\subsubsection{Benutzer, Gruppen und Events}

\begin{figure}[H]
    \centering
    \includegraphics[width=1.3\textwidth]{ldm_user_group_event.png}
    \caption{LDM Benutzer, Gruppen und Events}
    \label{fig:ldm1}
\end{figure}

Das logische Datenmodell bildet die zentralen Stammdaten des Systems EventHub ab.
Die Tabellen \texttt{users}, \texttt{groups} und \texttt{events} repräsentieren
Benutzer, Gruppen sowie geplante oder durchgeführte Events und stellen damit
die fachliche Grundlage der Anwendung dar.

Die Beziehung zwischen Benutzern und Gruppen wird über die Tabelle
\texttt{group\_memberships} modelliert. Dadurch können zusätzliche Informationen
wie Rolle und Status einer Mitgliedschaft (z.\,B. aktiv, gesperrt oder verlassen)
explizit gespeichert werden. Rollen werden separat verwaltet, um eine flexible
Zuordnung von Rechten innerhalb einer Gruppe zu ermöglichen.

Institutionen werden über die Tabelle \texttt{institutions} abgebildet. Die
Mitgliedschaft von Benutzern in einer Institution erfolgt über
\texttt{institution\_members}. Gruppen können optional einer Institution
zugeordnet sein, wodurch sowohl institutionelle als auch private Gruppen
unterstützt werden.

Events sind eindeutig einer Gruppe zugeordnet und enthalten neben zeitlichen
Angaben auch fachliche Attribute wie Mindestteilnehmerzahl, Mindestbudget
und Status. Eindeutige Felder wie E-Mail-Adresse, Benutzername, Beitrittscode
und öffentliche Event-URLs werden durch \texttt{unique}-Constraints abgesichert.
Alle Beziehungen sind über Fremdschlüssel eindeutig definiert.

\subsubsection{Notizen \& Umfragen}

\begin{figure}[H]
    \centering
    \includegraphics[width=1.3\textwidth]{ldm_notes_polls.png}
    \caption{LDM Notizen \& Umfragen}
    \label{fig:ldm2}
\end{figure}

Abbildung~\ref{fig:ldm2} zeigt das logische Datenmodell für gruppenbezogene
Notizen und Umfragen. Die Tabelle \texttt{notes} speichert alle Notizen als gemeinsame
Basisstruktur und unterscheidet über \texttt{note\_type} zwischen Freitext-, Termin-
und Ausgaben-Notizen.

Termin- und Ausgaben-spezifische Attribute werden in den Tabellen
\texttt{note\_appointments} bzw. \texttt{note\_expenses} abgelegt. Diese Tabellen
verwenden \texttt{note\_id} gleichzeitig als Primärschlüssel und Fremdschlüssel auf
\texttt{notes}, wodurch eine 1:1-Beziehung zwischen Basisnotiz und Spezialisierung
erzielt wird.

Umfragen werden über \texttt{polls} modelliert und sind eindeutig einer Gruppe sowie
einem Ersteller zugeordnet. Die Antwortmöglichkeiten sind in \texttt{poll\_options}
abgebildet; abgegebene Stimmen werden als Bewegungsdaten in \texttt{poll\_votes}
gespeichert. Dadurch sind Auswertungen (z.\,B. Ergebnisdarstellung) möglich, ohne
Umfragedaten nachträglich zu verändern.

\subsubsection{Finanzen \& Aufgaben}

\begin{figure}[H]
    \centering
    \includegraphics[width=1.3\textwidth]{ldm_payments_tasks.png}
    \caption{LDM Finanzen \& Aufgaben}
    \label{fig:ldm3}
\end{figure}

Das logische Datenmodell für Finanzen und Aufgaben bildet die zentralen
Bewegungsdaten im Kontext von Gruppen und Events ab. Finanzielle Transaktionen
werden über die Tabellen \texttt{payments} und \texttt{refunds} modelliert und
ermöglichen die Nachverfolgung von Einzahlungen sowie Rückzahlungen innerhalb
einer Gruppe.

Einzahlungen (\texttt{payments}) sind jeweils einem Benutzer und einer Gruppe
zugeordnet und enthalten neben dem Betrag auch Informationen zum Zahlungszeitpunkt,
Zahlungskanal und Status. Rückzahlungen (\texttt{refunds}) referenzieren ebenfalls
eine Gruppe und einen Benutzer als Empfänger; zusätzlich wird der auslösende
Benutzer (z.\,B. ein Organisator) separat gespeichert, um Verantwortlichkeiten
nachvollziehbar abzubilden.

Aufgaben werden über die Tabelle \texttt{tasks} modelliert und sind eindeutig
einem Event sowie einem Ersteller zugeordnet. Die konkrete Zuweisung von Aufgaben
an Benutzer erfolgt über \texttt{task\_assignments}. Dadurch können Aufgaben
mehreren Benutzern zugewiesen und Zustände wie Übernahme oder Erinnerung
zeitlich dokumentiert werden.

Durch die Trennung von Stammdaten (Aufgaben) und Bewegungsdaten (Zuweisungen,
Zahlungen, Rückzahlungen) bleibt das Datenmodell flexibel, erweiterbar und
konsistent.

\subsubsection{Dienstleister \& Abrechnung}

\begin{figure}[H]
    \centering
    \includegraphics[width=1.0\textwidth]{ldm_providers_invoices.png}
    \caption{LDM Dienstleister \& Abrechnung}
    \label{fig:ldm4}
\end{figure}

Das logische Datenmodell für Dienstleister und Abrechnung bildet den vollständigen
B2B-Prozess von der Anfrage über Angebot und Rechnung bis hin zu Mahnungen ab.
Dienstleister werden in der Tabelle \texttt{providers} verwaltet und können optional
mit einem Benutzerkonto verknüpft sein, sofern ein eigener Dienstleister-Login
existiert.

Anfragen an Dienstleister werden über \texttt{service\_requests} modelliert und
enthalten organisatorische sowie zeitliche Rahmenbedingungen eines geplanten Events.
Auf Basis einer Anfrage kann ein oder mehrere Angebote (\texttt{offers}) entstehen,
die Preisangaben und einen Angebotsstatus enthalten.

Wird ein Angebot angenommen, kann daraus eine Rechnung (\texttt{invoices}) erzeugt
werden. Rechnungen enthalten eindeutige Rechnungsnummern, Beträge, Zahlungsfristen
sowie Statusinformationen. Mahnungen werden in \texttt{reminders} separat gespeichert
und referenzieren jeweils eine Rechnung, wodurch mehrere Mahnstufen abgebildet werden
können.

Bewertungen von Dienstleistern erfolgen über \texttt{provider\_ratings} und sind
einem Benutzer sowie einem Dienstleister zugeordnet. Durch die klare Trennung der
einzelnen Prozessschritte bleibt das Datenmodell nachvollziehbar, erweiterbar und
fachlich konsistent.

\subsubsection{Ressourcenplanung}

\begin{figure}[H]
    \centering
    \includegraphics[width=1.0\textwidth]{ldm_resource_planning.png}
    \caption{LDM Ressourcenplanung}
    \label{fig:ldm5}
\end{figure}

Das logische Datenmodell zur Ressourcenplanung bildet Material- und Personaleinsatz
im Kontext von Dienstleisterangeboten ab. Materialien werden in \texttt{materials}
verwaltet und sind eindeutig einem Dienstleister (\texttt{provider\_id}) zugeordnet.
Über \texttt{material\_categories} können Materialien thematisch klassifiziert werden,
wobei Kategorienamen durch einen \texttt{unique}-Constraint eindeutig sind.

Konkrete Materialreservierungen für ein Angebot werden über \texttt{material\_bookings}
modelliert. Jede Buchung referenziert ein \texttt{offer} sowie ein \texttt{material} und
enthält Zeitraum und Menge, wodurch parallele Buchungen und wiederholte Ausleihen
abbildbar sind. Wartungsereignisse werden in \texttt{material\_maintenance} als eigene
Bewegungsdaten gespeichert und referenzieren das jeweilige Material, um Fälligkeiten
und durchgeführte Wartungen nachvollziehbar zu dokumentieren.

Mitarbeiter werden in \texttt{employees} verwaltet und ebenfalls einem Dienstleister
zugeordnet. Die Einsatzplanung erfolgt über \texttt{employee\_assignments}, die einen
Mitarbeiter mit einem Angebot verknüpft und Zeitraum sowie Einsatzstatus
(z.\,B. geplant, bestätigt, krank, Urlaub) speichert. Insgesamt ermöglicht das Modell
eine konsistente Planung und Nachverfolgung von Ressourcen pro Angebot.

\subsection{Übersicht der Teilmodelle und Querverweise}

\begin{figure}[H]
    \centering
    \includegraphics[width=1.0\textwidth]{ldm_connection.png}
    \caption{LDM Zusammenhang der LDM's}
    \label{fig:ldm6}
\end{figure}

Abbildung~\ref{fig:ldm6} zeigt eine kompakte Übersicht über den Zusammenhang
der fünf logischen Teilmodelle. Zentrale Entitäten sind \texttt{users}, \texttt{groups}
und \texttt{events}, auf denen die Community-Funktionen wie Notizen, Umfragen, Finanzen
und Aufgaben aufbauen.

Notizen und Umfragen sind gruppenbezogen modelliert, ebenso Einzahlungen und
Rückzahlungen. Aufgaben sind an Events gekoppelt und werden damit indirekt über die
zugehörige Gruppe kontextualisiert. Der Dienstleisterprozess verläuft von
\texttt{service\_requests} über \texttt{offers} zu \texttt{invoices} und ist über
\texttt{providers} mit Dienstleistern verknüpft.

Die Ressourcenplanung (Material und Mitarbeiter) ist dem Dienstleister zugeordnet und
wird über Angebote konkret für einzelne Aufträge/Events gebucht bzw. eingeplant.
Die Übersicht dient als Navigationshilfe und stellt sicher, dass die Teilmodelle
konsistent aufeinander referenzieren.

\subsection{CRUD-Matrix (User Stories $\leftrightarrow$ Entitäten)}

\subsection{CRUD-Matrix}

{\small
\setlength{\LTpre}{0pt}
\setlength{\LTpost}{0pt}

\begin{longtable}{ |p{3cm}|p{2.5cm}|p{2.5cm}|p{2.5cm}|p{2.5cm}| }
\caption{CRUD-Matrix}\\
\hline
\textbf{Entität (Tabelle)} & \textbf{Create (C)} & \textbf{Read (R)} & \textbf{Update (U)} & \textbf{Delete (D)} \\
\hline
\endfirsthead

\hline
\textbf{Entität (Tabelle)} & \textbf{Create (C)} & \textbf{Read (R)} & \textbf{Update (U)} & \textbf{Delete (D)} \\
\hline
\endhead

% --- Teilmodell 1: Benutzer/Gruppe/Event ---
users &
US-01 &
US-02, US-06 &
US-04, US-05, US-07, US-08, US-09 &
-- \\
\hline

institutions &
-- &
US-38 &
-- &
-- \\
\hline

institution\_members &
-- &
-- &
US-36, US-37 &
-- \\
\hline

groups &
US-10 &
US-14 &
US-15 (Code verwalten implizit), US-16 (Min. Teilnehmer/Budget) &
US-11 \\
\hline

group\_memberships &
US-10 (Organisator wird Mitglied), US-12 &
US-14 &
US-13 (Austritt als Statuswechsel), US-15 (sperren/entfernen), US-36, US-37 &
-- \\
\hline

events &
US-10 (Event im Rahmen der Gruppe anlegen), US-39 (öffentliche Seite implizit) &
US-39 &
US-16 &
US-11 (archivieren/löschen) \\
\hline

% --- Teilmodell 2: Notizen/Umfragen ---
notes &
US-17, US-21, US-27 &
US-20 &
US-18 &
US-19 \\
\hline

note\_appointments &
US-21 &
US-21, US-41 &
US-18 (wenn Termin-Notiz bearbeitet wird) &
US-19 (via Notiz löschen) \\
\hline

note\_expenses &
US-27 &
US-26, US-28 &
US-18 (wenn Ausgabe-Notiz bearbeitet wird) &
US-19 (via Notiz löschen) \\
\hline

polls &
US-22 &
US-24 &
-- &
-- \\
\hline

poll\_options &
US-22 &
US-24 &
-- &
-- \\
\hline

poll\_votes &
US-23 &
US-24 &
-- &
-- \\
\hline

% --- Teilmodell 3: Finanzen/Aufgaben ---
payments &
US-25 &
US-26, US-28 &
-- &
-- \\
\hline

refunds &
US-29 &
US-28 &
-- &
-- \\
\hline

tasks &
US-30 &
-- &
US-32 &
-- \\
\hline

task\_assignments &
US-31 &
-- &
US-33 (Erinnerung/Statusdaten) &
-- \\
\hline

% --- Medien/Kommentare (falls in eurem Modell) ---
media &
US-34 &
US-34 &
-- &
-- \\
\hline

comments &
US-35 &
-- &
-- &
-- \\
\hline

% --- Teilmodell 4: Dienstleister/Abrechnung ---
providers &
-- &
-- &
-- &
-- \\
\hline

service\_requests &
US-43 &
-- &
-- &
-- \\
\hline

offers &
US-44 &
-- &
-- &
-- \\
\hline

invoices &
US-45 &
-- &
-- &
-- \\
\hline

reminders &
US-46 &
-- &
-- &
-- \\
\hline

provider\_ratings &
-- &
-- &
-- &
-- \\
\hline

% --- Teilmodell 5: Ressourcenplanung ---
material\_categories &
-- &
-- &
-- &
-- \\
\hline

materials &
US-47 &
US-47, US-50 &
US-47 &
US-47 (falls ``entfernen'' vorgesehen) \\
\hline

material\_maintenance &
-- &
US-50 &
-- &
-- \\
\hline

material\_bookings &
US-48 &
-- &
-- &
-- \\
\hline

employees &
US-49 &
US-49 &
US-49 &
US-49 (falls ``entfernen'' vorgesehen) \\
\hline

employee\_assignments &
US-49 (Einsatzplanung implizit) &
US-49 &
US-49 &
-- \\
\hline

\end{longtable}
\noindent
\\
Die CRUD-Matrix zeigt die Zuordnung der User Stories zu den persistierten Entitäten.
Bewegungsdaten (z.\,B. Zahlungen, Abstimmungen, Buchungen) werden ausschließlich erzeugt
(Create) und gelesen (Read), jedoch nicht verändert oder gelöscht.


\section{Prozess-Sicht}

\subsection{Zustandsdiagramm: Event}

\begin{figure}[H]
    \centering
    \includegraphics[width=1.3\textwidth]{Zustandsdiagramm_Event.png}
    \caption{Zustandsdiagramm\_Event.png}
    \label{fig:zustandsdiagramm_event}
\end{figure}

Das Zustandsdiagramm beschreibt den Lebenszyklus eines Events von der initialen
Erstellung bis zur Archivierung oder Löschung. Nach dem Anlegen befindet sich ein
Event zunächst im Zustand \textit{Entwurf} und kann geplant, geändert oder verworfen
werden.

Im Zustand \textit{Geplant} werden fachliche Bedingungen wie Mindestteilnehmerzahl
und Mindestbudget geprüft. Sind diese erfüllt, wird das Event freigeschaltet und kann
zum geplanten Zeitpunkt oder manuell in den Zustand \textit{Aktiv} übergehen. Nach
Abschluss der Veranstaltung wechselt das Event in den Zustand \textit{Abgeschlossen}
und kann anschließend archiviert oder gemäß definierter Regeln gelöscht werden.

Das Diagramm stellt sicher, dass alle fachlich zulässigen Zustandsübergänge eindeutig
definiert sind und der Eventstatus jederzeit konsistent ist.

\subsection{Zustandsdiagramm: Rechnung}

\begin{figure}[H]
    \centering
    \includegraphics[width=1.3\textwidth]{Zustandsdiagramm_Rechnung.png}
    \caption{Zustandsdiagramm Rechnung}
    \label{fig:zustandsdiagramm_rechnung}
\end{figure}

Das Zustandsdiagramm für Rechnungen modelliert den fachlichen Ablauf von der Erstellung
einer Rechnung bis zu deren Abschluss. Eine Rechnung wird zunächst im Zustand
\textit{offen} angelegt und kann durch Zahlung in den Zustand \textit{bezahlt}
übergehen.

Erfolgt innerhalb der definierten Frist keine Zahlung, wechselt die Rechnung in den
Zustand \textit{überfällig}. In diesem Zustand können Mahnungen erzeugt werden, wodurch
der Status entsprechend aktualisiert wird. Nach vollständigem Zahlungseingang gilt
die Rechnung als abgeschlossen.

Durch die Modellierung der Zustände wird sichergestellt, dass Rechnungen nachvollziehbar
bearbeitet werden und finanzielle Prozesse transparent und regelkonform ablaufen.

\subsection{Sequenzdiagramm: Gruppe per Beitrittscode beitreten}

\begin{figure}[H]
    \centering
    \includegraphics[width=1.3\textwidth]{Sequenzdiagramm_Gruppebeitreten.png}
    \caption{Sequenzdiagramm Gruppe per Beitritscode beitreten}
    \label{fig:seq-join-group}
\end{figure}

Abbildung~\ref{fig:seq-join-group} zeigt den Ablauf eines Gruppenbeitritts über einen
Beitrittscode. Der Benutzer initiiert den Vorgang über den Web- oder Mobile-Client,
welcher einen REST-Request an die Server-Schnittstelle sendet.

Die Server-Schicht prüft die Gültigkeit des Beitrittscodes, den Status der Gruppe
sowie bestehende Mitgliedschaften. Ist der Beitritt zulässig, wird eine neue
Mitgliedschaft mit der Rolle \textit{Mitglied} angelegt und persistiert. Andernfalls
wird eine geeignete Fehlermeldung an den Client zurückgegeben.

Das Diagramm verdeutlicht die Trennung von Präsentation, Geschäftslogik und
Datenhaltung sowie die serverseitige Durchsetzung fachlicher Regeln.

\subsection{Sequenzdiagramm: Einzahlung erfassen}

\begin{figure}[H]
    \centering
    \includegraphics[width=1.3\textwidth]{Sequenzdiagramm_Einzahlungerfassen.png}
    \caption{Sequenzdiagramm Einzahlung erfassen}
    \label{fig:seq-payment}
\end{figure}

Abbildung~\ref{fig:seq-payment} zeigt den Ablauf zur Erfassung einer Einzahlung durch
ein Gruppenmitglied. Der Benutzer gibt den Betrag im Client ein, woraufhin ein
authentifizierter REST-Request an die Server-Schnittstelle gesendet wird.

Die Server-Schicht prüft zunächst, ob der Benutzer aktives Mitglied der angegebenen
Gruppe ist. Ist dies der Fall, wird die Einzahlung als Bewegungsdatum in der Datenbank
persistiert und dem Client eine erfolgreiche Bestätigung zurückgegeben. Andernfalls
wird der Vorgang mit einem Fehlerstatus abgelehnt.

Das Sequenzdiagramm verdeutlicht die serverseitige Durchsetzung fachlicher Regeln
sowie die klare Trennung zwischen Client, Geschäftslogik und Datenhaltung.

\section{Entwicklungs-Sicht}

\subsection{Drei-Schichten-Architektur (Client – Server – Datenbank)}

EventHub wird als klassische Drei-Schichten-Architektur umgesetzt. Ziel ist eine klare
Trennung von Verantwortlichkeiten zwischen Darstellung, Geschäftslogik und Datenhaltung.
Dadurch werden Wartbarkeit, Skalierbarkeit und Sicherheit verbessert.

\subsection{Client-Schicht (Präsentation)}

Die Client-Schicht umfasst die Webanwendung (Browser) sowie die Mobile App. Sie ist
verantwortlich für:
\begin{itemize}
  \item Benutzerinteraktion (Formulare, Navigation, Validierung auf UI-Ebene)
  \item Darstellung von Gruppen, Events, Notizen, Umfragen, Finanzen und Aufgaben
  \item Aufruf der Backend-Schnittstellen über HTTPS (REST)
  \item Token-basierte Authentifizierung (z.\,B. Speicherung eines Access Tokens)
\end{itemize}

\noindent
\textbf{Wichtig:} Fachlogik (z.\,B. Rollenprüfung oder Finanzberechnungen) liegt nicht im
Client, sondern ausschließlich in der Server-Schicht, um eine Umgehung fachlicher Regeln
zu verhindern.

\subsection{Server-Schicht (Applikation / Geschäftslogik)}

Die Server-Schicht stellt die zentrale REST-API bereit und enthält die vollständige
Geschäftslogik der Anwendung. Sie übernimmt unter anderem:
\begin{itemize}
  \item Authentifizierung (Login, Zwei-Faktor-Authentifizierung) und Autorisierung
        (Rollen und Rechte wie Organisator, Mitglied, Institutionsadministrator,
        Dienstleister)
  \item Validierung fachlicher Regeln (z.\,B. Beitrittscode, Mindestteilnehmer,
        Mindestbudget, Sperrstatus)
  \item Orchestrierung komplexer Use Cases (z.\,B. Anfrage $\rightarrow$ Angebot
        $\rightarrow$ Rechnung $\rightarrow$ Mahnung)
  \item Aggregationen und Berichte (z.\,B. Kassenübersicht aus Einzahlungen, Ausgaben
        und Rückzahlungen)
  \item Integration externer Dienste (z.\,B. Kalendersynchronisation oder
        Benachrichtigungskanäle) über Adapter oder Services
\end{itemize}

Die Server-Schicht kapselt sämtliche Datenzugriffe (z.\,B. über Repository- oder
DAO-Konzepte) und gibt nach außen ausschließlich kontrollierte Datenobjekte
(DTOs/JSON) zurück.

\subsection{Datenbank-Schicht (Persistenz)}

Die Datenbank-Schicht ist für die persistente Speicherung aller relevanten Daten
verantwortlich. Dazu zählen unter anderem:
\begin{itemize}
  \item Stammdaten: Benutzer, Gruppen, Events, Dienstleister, Material, Mitarbeiter
  \item Bewegungsdaten: Einzahlungen, Abstimmungen, Buchungen, Einsätze, Mahnungen
  \item Status- und Historieninformationen (z.\,B. Mitgliedschaft aktiv, gesperrt oder
        verlassen)
\end{itemize}

Zugriffe auf die Datenbank erfolgen ausschließlich über die Server-Schicht. Ein direkter
Datenbankzugriff vom Client ist nicht möglich. Dadurch können Rechteprüfungen,
Datenschutzmaßnahmen und Konsistenzregeln zentral durchgesetzt werden.

\subsection{Datenfluss}

Der typische Datenfluss innerhalb der Drei-Schichten-Architektur gestaltet sich wie
folgt:
\begin{enumerate}
  \item Der Client sendet einen Request (HTTPS/REST) an den Server, inklusive
        Authentifizierungs-Token.
  \item Der Server prüft Token und Berechtigungen und führt die entsprechende
        Geschäftslogik aus.
  \item Der Server liest oder schreibt Daten in der Datenbank.
  \item Der Server antwortet mit einem JSON-Ergebnis, das vom Client dargestellt wird.
\end{enumerate}

\subsection{Vorteile der Architektur}

Die gewählte Drei-Schichten-Architektur bietet für EventHub folgende Vorteile:
\begin{itemize}
  \item \textbf{Sicherheit:} Rechte- und DSGVO-Regeln werden zentral im Server
        durchgesetzt und können nicht clientseitig umgangen werden.
  \item \textbf{Wartbarkeit:} Benutzeroberfläche, Geschäftslogik und Datenhaltung
        können unabhängig voneinander weiterentwickelt werden.
  \item \textbf{Skalierbarkeit:} Die Server-Schicht kann horizontal skaliert werden,
        während die Datenbank separat optimiert werden kann.
  \item \textbf{Portabilität:} Web- und Mobile-Clients nutzen dieselbe REST-API.
\end{itemize}

\subsection{Komponentendiagramm}

\begin{figure}[H]
    \centering
    \includegraphics[width=0.5\textwidth]{Komponentendiagramm.png}
    \caption{Komponentendiagramm}
    \label{fig:Komponentendiagramm}
\end{figure}

Abbildung~\ref{fig:Komponentendiagramm} visualisiert die Drei-Schichten-Architektur von
EventHub als Komponentendiagramm. Die Client-Schicht (Web Client und Mobile App)
kommuniziert ausschließlich über HTTPS/JSON mit der REST-API der Server-Schicht.

In der Server-Schicht kapselt die \textit{Business Logic} die fachlichen Regeln und
Use-Case-Abläufe, während \textit{Authentification \& Authorization} die Authentifizierung und
rollenbasierte Zugriffsprüfung zentral durchsetzt. Persistente Daten werden in der
Datenbank-Schicht gespeichert; der Zugriff darauf erfolgt ausschließlich serverseitig,
wodurch Konsistenz- und Sicherheitsanforderungen zentral kontrolliert werden können.

\subsection{REST-API-Spezifikation}

\subsubsection{API – Konventionen}

\begin{itemize}
  \item \textbf{Base URL:} \texttt{/api/v1}
  \item \textbf{Datenformat:} JSON (UTF-8)
  \item \textbf{Authentifizierung:} Token-basiert (Authorization: Bearer Token)
  \item \textbf{Autorisierung:} Rollen- und Rechteprüfung serverseitig
  \item \textbf{Statuscodes:} 200 OK, 201 Created, 204 No Content, 400 Bad Request,
        401 Unauthorized, 403 Forbidden, 404 Not Found, 409 Conflict
\end{itemize}

\subsubsection{Ressourcen und Endpunkte}

\paragraph{Authentifizierung und Benutzer}

\begin{longtable}{|p{5cm}|p{2cm}|p{7cm}|}
\hline
\textbf{Endpoint} & \textbf{Methode} & \textbf{Beschreibung} \\
\hline
\endfirsthead
\hline
\textbf{Endpoint} & \textbf{Methode} & \textbf{Beschreibung} \\
\hline
\endhead

/auth/register & POST & Benutzer registrieren (US-01) \\
\hline
/auth/login & POST & Benutzer anmelden (US-02) \\
\hline
/auth/2fa/verify & POST & Zwei-Faktor-Authentifizierung bestätigen \\
\hline
/auth/logout & POST & Benutzer abmelden \\
\hline
/users/me & GET & Eigenes Benutzerprofil abrufen (US-06) \\
\hline
/users/me & PATCH & Profil bearbeiten (US-04, US-07, US-08, US-09) \\
\hline
/users/me/privacy-consent & POST & DSGVO-Zustimmung speichern \\
\hline
\end{longtable}

\paragraph{Gruppen und Mitgliedschaften}

\begin{longtable}{|p{5cm}|p{2cm}|p{7cm}|}
\hline
\textbf{Endpoint} & \textbf{Methode} & \textbf{Beschreibung} \\
\hline
\endfirsthead
\hline
\textbf{Endpoint} & \textbf{Methode} & \textbf{Beschreibung} \\
\hline
\endhead

/groups & POST & Gruppe anlegen (US-10) \\
\hline
/groups & GET & Eigene Gruppen anzeigen (US-14) \\
\hline
/groups/{groupId} & GET & Gruppendetails anzeigen \\
\hline
/groups/{groupId} & DELETE & Gruppe löschen/archivieren (US-11) \\
\hline
/groups/{groupId}/join-code & POST & Beitrittscode verwalten (US-15) \\
\hline
/groups/{groupId}/members/join & POST & Gruppe per Code beitreten (US-12) \\
\hline
/groups/{groupId}/members/{userId} & PATCH & Mitglied sperren/Rolle ändern (US-15) \\
\hline
/groups/{groupId}/leave & POST & Gruppe verlassen (US-13) \\
\hline
\end{longtable}

\paragraph{Events}

\begin{longtable}{|p{5cm}|p{2cm}|p{7cm}|}
\hline
\textbf{Endpoint} & \textbf{Methode} & \textbf{Beschreibung} \\
\hline
\endfirsthead
\hline
\textbf{Endpoint} & \textbf{Methode} & \textbf{Beschreibung} \\
\hline
\endhead

/groups/{groupId}/events & POST & Event anlegen \\
\hline
/groups/{groupId}/events & GET & Events einer Gruppe anzeigen \\
\hline
/events/{eventId} & GET & Eventdetails anzeigen \\
\hline
/events/{eventId} & PATCH & Event bearbeiten (US-16) \\
\hline
/events/{eventId}/archive & POST & Event archivieren \\
\hline
/events/{eventId}/public & GET & Öffentliche Eventseite (US-39) \\
\hline
\end{longtable}

\paragraph{Notizen, Umfragen und Finanzen}

\begin{longtable}{|p{5cm}|p{2cm}|p{7cm}|}
\hline
\textbf{Endpoint} & \textbf{Methode} & \textbf{Beschreibung} \\
\hline
\endfirsthead
\hline
\textbf{Endpoint} & \textbf{Methode} & \textbf{Beschreibung} \\
\hline
\endhead

/groups/{groupId}/notes & POST & Notiz anlegen (US-17, US-21, US-27) \\
\hline
/groups/{groupId}/notes & GET & Notizen anzeigen (US-20) \\
\hline
/notes/{noteId} & PATCH & Notiz bearbeiten (US-18) \\
\hline
/notes/{noteId} & DELETE & Notiz löschen (US-19) \\
\hline
/groups/{groupId}/polls & POST & Umfrage anlegen (US-22) \\
\hline
/polls/{pollId}/vote & POST & Abstimmen (US-23) \\
\hline
/groups/{groupId}/payments & POST & Einzahlung erfassen (US-25) \\
\hline
/groups/{groupId}/finance/summary & GET & Kassenübersicht (US-28) \\
\hline
\end{longtable}

\noindent
\\
Die REST-API stellt alle Systemfunktionen als versionierte Ressourcen bereit.
Die Kommunikation erfolgt ausschließlich über HTTPS mit JSON als Austauschformat.
Geschäftslogik sowie Sicherheits- und Datenschutzregeln werden serverseitig durchgesetzt.


\section{Verteilungs-Sicht}

\subsection{UML-Verteilungsdiagramm}

\begin{figure}[H]
    \centering
    \includegraphics[width=1.2\textwidth]{Verteilungsdiagramm.png}
    \caption{Verteilungsdiagramm}
    \label{fig:deployment}
\end{figure}

Abbildung~\ref{fig:deployment} zeigt die physische Verteilung der Systemkomponenten
von EventHub zur Laufzeit. Web-Client (Browser) und Mobile App laufen auf
Client-Geräten und kommunizieren ausschließlich über HTTPS/REST mit dem
Backend-Server.

Der Backend-Server hostet die REST-API sowie zentrale Komponenten für
Geschäftslogik und Authentifizierung/Autorisierung. Persistente Daten werden auf
einem separaten Datenbank-Server gespeichert, der ausschließlich vom Backend
angesprochen wird.

Zusätzlich sind optionale externe Dienste angebunden, darunter ein
Benachrichtigungsdienst (E-Mail/SMS/Push), ein Kalenderdienst sowie ein
Objektspeicher für Medieninhalte. Diese werden über gesicherte HTTPS-Schnittstellen
vom Backend aus integriert.

\subsection{Systemlandschaft und externe Dienste}


\section{Sicherheit und Datenschutz}

\subsection{STRIDE-Sicherheitsanalyse}

Die Sicherheitsanalyse des Systems erfolgt anhand des STRIDE-Modells. Betrachtet wird
der Systemkontext bestehend aus Web-/Mobile-Client, REST-API (Backend) und Datenbank.
\\
\begin{longtable}{|p{2cm}|p{4cm}|p{4cm}|p{5cm}|}
\hline
\textbf{STRIDE} & \textbf{Risiko} & \textbf{Beispiel} & \textbf{Gegenmaßnahmen} \\
\hline
\endfirsthead
\hline
\textbf{STRIDE} & \textbf{Risiko} & \textbf{Beispiel} & \textbf{Gegenmaßnahmen} \\
\hline
\endhead

S (Spoofing) &
Identitätsvortäuschung &
Account-Übernahme durch gestohlene Zugangsdaten &
Passwort-Hashing (Argon2/bcrypt), HTTPS, 2FA, Rate-Limiting, Token mit begrenzter Laufzeit \\
\hline

T (Tampering) &
Manipulation von Daten &
Manipulierte Requests mit fremden IDs &
Serverseitige Validierung, Autorisierungsprüfungen, Prepared Statements, Ownership-Checks \\
\hline

R (Repudiation) &
Abstreitbarkeit von Aktionen &
Benutzer bestreitet Zahlung oder Sperre &
Audit-Logging (wer/was/wann), serverseitige Zeitstempel, Request-IDs \\
\hline

I (Information Disclosure) &
Informationsleck &
Unbefugter Zugriff auf Gruppen- oder Profildaten &
RBAC, Datenminimierung, sichere Fehlerausgaben, Verschlüsselung in Transit und at Rest \\
\hline

D (Denial of Service) &
Dienstverweigerung &
Brute-Force-Login oder API-Spam &
Rate-Limits, Timeouts, Paging, WAF/Reverse-Proxy \\
\hline

E (Elevation of Privilege) &
Rechteausweitung &
Mitglied wird unberechtigt Organisator &
Serverseitige Rollenprüfung, getrennte Admin-Endpunkte, gültige Rollen-Constraints \\
\hline
\end{longtable}

\subsection{Schutzmaßnahmen}

\subsubsection{Authentifizierung}
Die Authentifizierung erfolgt über ein tokenbasiertes Verfahren (Bearer Token).
Passwörter werden ausschließlich gehasht und gesalzen gespeichert. Optional kann eine
Zwei-Faktor-Authentifizierung (2FA) aktiviert werden.

\subsubsection{Autorisierung}
Die Autorisierung erfolgt serverseitig über rollenbasierte Zugriffskontrollen (RBAC).
Zusätzlich wird bei jedem Zugriff geprüft, ob der Benutzer zur angeforderten Ressource
(z.\,B. Gruppe oder Event) gehört.

\subsubsection{Eingabevalidierung und Datenintegrität}
Alle Eingaben werden serverseitig validiert. Datenbankzugriffe erfolgen ausschließlich
über vorbereitete Statements oder ORM-Mechanismen, um Manipulationen zu verhindern.

\subsubsection{Transport- und Datensicherheit}
Die gesamte Kommunikation erfolgt über TLS (HTTPS). Persistente Daten werden verschlüsselt
gespeichert oder auf verschlüsselten Datenträgern abgelegt. Backups werden ebenfalls
verschlüsselt gespeichert.

\subsubsection{Logging und Monitoring}
Sicherheitsrelevante Aktionen wie Login-Versuche, Rollenänderungen oder Zahlungen werden
auditierbar protokolliert. Monitoring-Mechanismen erkennen ungewöhnliche Zugriffsmuster.

\subsection{Datenschutz und DSGVO-Umsetzung}

\subsubsection{Zugriffsschutz (Privacy by Design)}
Der Zugriff auf personenbezogene Daten ist ausschließlich berechtigten Benutzern möglich.
Öffentlich zugängliche Inhalte zeigen nur explizit freigegebene Informationen.

\subsubsection{Datenminimierung}
Es werden nur Daten erhoben und gespeichert, die für den Betrieb der Anwendung erforderlich
sind. Sensible Informationen wie Passwörter werden niemals im Klartext gespeichert.

\subsubsection{Einwilligung und Transparenz}
Die Zustimmung zur Datenschutzerklärung wird mit Zeitstempel und Versionsnummer
(\texttt{privacy\_accepted\_at}, \texttt{privacy\_version}) gespeichert und ist nachvollziehbar.

\subsubsection{Löschkonzept und Aufbewahrung}
Benutzer können ihr Konto löschen lassen. Personenbezogene Daten werden dabei gelöscht oder
anonymisiert. Beim Austritt aus Gruppen werden zugehörige Einzahlungen gemäß fachlicher
Vorgabe entfernt oder logisch gelöscht (Soft-Delete).

\subsubsection{Betroffenenrechte}
Das System unterstützt die Rechte auf Auskunft, Berichtigung und Löschung personenbezogener
Daten. Ein Export der eigenen Daten kann in strukturierter Form bereitgestellt werden.


\section{Fazit und Ausblick (eventuell entfernen)}


\section{AB HIER ENDET DAS OFFIZIELLE INHALTSVERZEICHNISS}


\end{document}