\documentclass[12pt,a4paper]{article}
\usepackage[utf8]{inputenc}
\usepackage{enumitem}
\usepackage{tocloft} % Optional: Falls du das Inhaltsverzeichnis schöner formatieren willst
\usepackage{graphicx} % Required for inserting images
\usepackage{longtable}
\usepackage{array}
\usepackage{parskip}
\usepackage{float}
\usepackage[T1]{fontenc}
\usepackage[utf8]{inputenc}
\usepackage[ngerman]{babel}

% --- Sprache/Encoding --- 
\usepackage[T1]{fontenc}
\usepackage[utf8]{inputenc}
\usepackage[ngerman]{babel}

% --- Layout/Typo ---
\usepackage{geometry}
\geometry{margin=2.5cm}
\usepackage{microtype}

% --- Tabellen ---
\usepackage{longtable}
\usepackage{tabularx}
\usepackage{array}
\usepackage{booktabs}
\usepackage{ltablex} % kombiniert longtable + tabularx
\keepXColumns

\usepackage[hidelinks]{hyperref}

% --- Prompt-/Codeblöcke (automatische Zeilenumbrüche) ---
\usepackage{fancyvrb}
\usepackage{fvextra}

\DefineVerbatimEnvironment{prompt}{Verbatim}{
  breaklines=true,
  breakanywhere=true,
  breaksymbolleft={},   % ← Pfeil AUS
  breaksymbolright={},  % ← Sicherheitshalber auch AUS
  fontsize=\small
}


\newcolumntype{Y}{>{\raggedright\arraybackslash}X}
\newcommand{\prio}[1]{\textbf{#1}}


\title{\textbf{EventHub}}
\author{\textbf{Felix Franke 1318821}\\ \textbf{Markus Wurms 50106907}}
\date{\textbf{6. Februar 2026}}

\begin{document}

\maketitle

\newpage

\tableofcontents

\newpage

\vspace{\baselineskip}  % Zeilenumbruch der aktuellen Schriftgröße

\newpage

\section{Fachliche Grundlagen und Anforderungen}

\subsection{Zielsetzung des Softwareprodukts}
Die Zielsetzung des Softwareprodukts ist es, eine zentrale Plattform zur Planung, Organisation, Durchführung und Nachbereitung von Veranstaltungen zu erstellen. Dazu sollen viele verschiedene Tools zusammengelegt werden, wie zum Beispiel verschiedene Messenger, E-Mail, Excel und Ticketsysteme. Das System sammelt alle relevanten Informationen an einem Ort bzw. auf einer Plattform.

Das System soll Transparenz über Termine, Aufgaben, Finanzen und Teilnehmer bieten. Dabei soll das Softwareprodukt sowohl für private als auch kommerzielle Nutzung dienen. Weitere Anforderungen sind die Endgeräteunabhängigkeit und die Skalierbarkeit des Produktes. Das Produkt bietet eine Kombination aus Eventplanung, Finanzverwaltung und Dienstleisterintegration.

\subsubsection{Produktname und Begründung}
Wir nennen das Softwareprodukt EventHub. Der Name ist leicht merkbar und simpel, spiegelt den Zweck der Software gut wieder und ist international verständlich. Außerdem ist er nicht zu technisch, aber gleichzeitig professionell.

% ---------------------------------------------------------
\subsection{Akteure und Rollen}

\subsubsection{Nicht registrierte / unangemeldete Akteure}
Ein nicht angemeldeter Akteur hat folgende Optionen:
\begin{itemize}[leftmargin=*,nosep]
  \item Registrierung durchführen
  \item Login durchführen
  \item öffentliche Event- und Dienstleisterseiten aufrufen
\end{itemize}

\subsubsection{Registrierte Akteure}

\paragraph{Privatnutzer/Teilnehmer}
Das sind die Nutzer, die Events beitreten und daran teilnehmen. Sie können:
\begin{itemize}[leftmargin=*,nosep]
  \item sich registrieren/anmelden
  \item Gruppen beitreten/verlassen
  \item Notizen, Termine, Umfragen erstellen und daran teilnehmen
  \item abstimmen \& kommentieren
  \item Einzahlungen vornehmen
  \item Aufgaben übernehmen
  \item Medien hochladen
  \item Kalender synchronisieren
\end{itemize}

\paragraph{Gruppenmitglied}
Ein Gruppenmitglied kann:
\begin{itemize}[leftmargin=*,nosep]
  \item Inhalte einer Gruppe lesen
  \item Notizen erstellen/bearbeiten/löschen
  \item an Umfragen teilnehmen
  \item Aufgaben übernehmen
  \item Einzahlungen vornehmen
  \item Multimedia hochladen
  \item Gruppe verlassen
\end{itemize}

\paragraph{Organisator/Gruppenleiter}
\begin{itemize}[leftmargin=*,nosep]
  \item Gruppe/Event erstellen und löschen
  \item Mitglieder einladen/entfernen/sperren
  \item Beitrittscode verwalten
  \item Mindestteilnehmerzahl festlegen
  \item Mindestbudget festlegen
  \item Aufgaben zuweisen
  \item Rückzahlungen veranlassen
  \item Eventseite verwalten
  \item Übersicht über Finanzen
  \item Moderationsmöglichkeiten
\end{itemize}

\paragraph{Institutionsadministrator}
Das können zum Beispiel Lehrer oder Firmenverantwortliche sein:
\begin{itemize}[leftmargin=*,nosep]
  \item mehrere Events verwalten
  \item Rollen und Berechtigungen vergeben
  \item Benutzerkonten innerhalb der Institution verwalten
  \item Übersicht über alle Veranstaltungen der Organisation
  \item Nutzungslimits überwachen
  \item Abrechnungen/Events archivieren
\end{itemize}

\paragraph{Institutionsmitglied}
\begin{itemize}[leftmargin=*,nosep]
  \item meist nur Leserechte
  \item Teilnahme an Events
  \item Aufgaben ausführen
\end{itemize}

\paragraph{Veranstaltungs-Dienstleister}
\begin{itemize}[leftmargin=*,nosep]
  \item eigenes Profil/Homepage verwalten
  \item Eventanfragen empfangen
  \item Angebote erstellen
  \item Rechnungen generieren
  \item Ressourcen planen
  \item Material und Personal verwalten
  \item Budgets kalkulieren
  \item durch Kunden bewerten lassen
\end{itemize}

\paragraph{Mitarbeiter eines Dienstleisters}
\begin{itemize}[leftmargin=*,nosep]
  \item persönliche Verfügbarkeit pflegen
  \item Einsätze/Arbeitspläne einsehen
  \item Benachrichtigungen bei Zuweisung erhalten
  \item Qualifikationen verwalten
\end{itemize}

\paragraph{Kunde/Veranstalter (extern)}
\begin{itemize}[leftmargin=*,nosep]
  \item Event anfragen
  \item Budget angeben
  \item Angebote erhalten
  \item Angebote annehmen und ablehnen
  \item Rechnungen bezahlen
  \item Dienstleister bewerten
\end{itemize}

\subsubsection{Sekundäre Akteure}

\paragraph{Externe Systeme}
\begin{itemize}[leftmargin=*,nosep]
  \item Kalenderdienste (z.\,B. Outlook, Google Kalender)
  \item Kommunikationsdienste (E-Mail, SMS, Messenger)
  \item Zahlungsdienste
  \item Ticketing-Systeme
  \item Datenaustausch über APIs
\end{itemize}

\paragraph{Systemadministrator}
\begin{itemize}[leftmargin=*,nosep]
  \item Benutzer sperren/verwalten
  \item Systemkonfiguration
  \item Backups/Wartung
\end{itemize}

% ---------------------------------------------------------
\subsection{Glossar der Fachdomäne}

\subsubsection{Event}
Eine geplante Veranstaltung innerhalb des Systems, die organisatorische Informationen wie Termine, Aufgaben, Finanzen, Teilnehmer und Dokumentation bündelt.

\subsubsection{Gruppe}
Eine Menge von Benutzern, die gemeinsam ein Event planen oder daran teilnehmen. Gruppen dienen als organisatorische Einheit zur Zusammenarbeit.

\subsubsection{Teilnehmer (Privatnutzer)}
Ein registrierter Benutzer, der einer oder mehreren Gruppen beitritt und aktiv an Events teilnimmt, beispielsweise durch Abstimmungen, Aufgabenübernahme oder Einzahlungen.

\subsubsection{Organisator (Gruppenleiter)}
Ein Benutzer mit erweiterten Rechten innerhalb einer Gruppe, der Events erstellt, Mitglieder verwaltet und organisatorische Entscheidungen trifft.

\subsubsection{Dienstleister}
Ein externer Anbieter von Leistungen oder Ressourcen für Events, der Anfragen empfängt, Angebote erstellt, Rechnungen ausstellt und Material sowie Personal plant.

\subsubsection{Aufgabe}
Ein konkretes To-do innerhalb eines Events, das einer oder mehreren Personen zugewiesen werden kann und einen Status wie offen, in Bearbeitung oder erledigt besitzt.

\subsubsection{Notiz}
Ein freier Informationseintrag innerhalb eines Events zur Dokumentation von Ideen, Absprachen oder Hinweisen.

\subsubsection{Umfrage}
Ein strukturiertes Abstimmungswerkzeug mit mehreren Antwortoptionen, das zur Entscheidungsfindung innerhalb einer Gruppe genutzt wird.

\subsubsection{Budget}
Der für ein Event verfügbare Geldrahmen, bestehend aus Einzahlungen und Ausgaben, der zur Planung und Kontrolle der Finanzierung dient.

\subsubsection{Rolle}
Eine Sammlung von Berechtigungen, die festlegt, welche Funktionen ein Benutzer im System ausführen darf, beispielsweise Lesen, Bearbeiten oder Administrieren.

\subsubsection{Institution}
Eine organisatorische Einheit wie Schule oder Unternehmen, die mehrere Events verwaltet und eigene Administratoren besitzt.

\subsubsection{Institutionsadministrator}
Ein Benutzer mit globalen Verwaltungsrechten innerhalb einer Institution, der Rollen vergibt und Benutzerkonten verwaltet.

\subsubsection{Ressource}
Materielle oder personelle Mittel (z.\,B. Geräte oder Mitarbeiter), die einem Event zugeordnet und eingeplant werden.

% =========================================================
\section{Funktionale Anforderungen}

\subsection{Funktionale Gruppen Übersicht}
\begin{itemize}[leftmargin=*,nosep]
  \item FG1 Account und Authentifizierung
  \item FG2 Profil und Einstellungen
  \item FG3 Gruppen- und Eventmanagement
  \item FG4 Notizen und Umfragen
  \item FG5 Finanzen
  \item FG6 Aufgabenmanagement
  \item FG7 Medien und Dokumentation
  \item FG8 Rollen- und Rechtesystem
  \item FG9 Öffentliche Eventseite und QR-Code
  \item FG10 Integration und Benachrichtigungen
  \item FG11 Dienstleister und Anfragen
  \item FG12 Angebote, Rechnungen und Mahnungen
  \item FG13 Ressourcenplanung
\end{itemize}

% ---------- Tabellen-Template ----------
\newcommand{\ustable}[2]{%
\subsection{#1}
\begin{longtable}{p{2.3cm}p{2.1cm}p{2.7cm}p{2.7cm}p{3.4cm}p{1.2cm}}
\toprule
\textbf{Name/ID} & \textbf{In meiner Rolle als...} & \textbf{...möchte ich...} & \textbf{..., so dass...} & \textbf{Akzeptiert, wenn...} & \textbf{Priorität}\\
\midrule
\endfirsthead
\toprule
\textbf{Name/ID} & \textbf{In meiner Rolle als...} & \textbf{...möchte ich...} & \textbf{..., so dass...} & \textbf{Akzeptiert, wenn...} & \textbf{Priorität}\\
\midrule
\endhead
#2
\bottomrule
\end{longtable}
}

\ustable{FG1 Account und Authentifizierung}{
US-01 Registrierung & Gast & mich registrieren & ich das System nutzen kann & Pflichtfelder sind ausgefüllt und Account wird erstellt & \prio{Muss}\\
US-02 Login & Benutzer & mich anmelden & ich Zugriff auf meine Daten habe & korrekte Zugangsdaten erlauben Anmeldung & \prio{Muss}\\
US-03 Logout & Benutzer & mich abmelden & niemand mein Konto missbraucht & Sitzung wird beendet & \prio{Muss}\\
US-04 Passwort zurücksetzen & Benutzer & mein Passwort zurücksetzen & ich wieder Zugriff bekomme & Reset-Link oder Code ermöglicht neues Passwort & \prio{Muss}\\
US-05 2FA aktivieren & Benutzer & Zwei-Faktor-Authentifizierung nutzen & mein Account sicherer ist & zusätzlicher Faktor wird beim Login geprüft & \prio{Sollte}\\
}

\ustable{FG2 Profil und Einstellungen}{
US-06 Profil anzeigen & Benutzer & mein Profil sehen & ich meine Daten prüfen kann & gespeicherte Daten werden angezeigt & \prio{Muss}\\
US-07 Profil bearbeiten & Benutzer & meine Daten ändern & sie aktuell bleiben & Änderungen werden gespeichert & \prio{Muss}\\
US-08 Kommunikationsweg wählen & Benutzer & bevorzugte Benachrichtigungen einstellen & ich passende Nachrichten erhalte & gewählter Kanal wird gespeichert & \prio{Sollte}\\
US-09 Sprache wählen & Benutzer & die Sprache ändern & ich die Oberfläche verstehe & UI-Texte werden angepasst & \prio{Sollte}\\
}

\ustable{FG3 Gruppen- und Eventmanagement}{
US-10 Gruppe erstellen & Organisator & eine Gruppe anlegen & ich ein Event planen kann & Name und Beschreibung werden gespeichert & \prio{Muss}\\
US-11 Gruppe löschen & Organisator & meine Gruppe löschen & alte Events entfernt werden & Gruppe wird archiviert oder gelöscht & \prio{Muss}\\
US-12 Gruppe beitreten & Benutzer & einer Gruppe beitreten & ich teilnehmen kann & gültiger Beitrittscode erlaubt Beitritt & \prio{Muss}\\
US-13 Gruppe verlassen & Benutzer & eine Gruppe verlassen & ich nicht mehr beteiligt bin & Mitgliedschaft endet & \prio{Muss}\\
US-14 Mitglieder anzeigen & Benutzer & alle Mitglieder sehen & ich Überblick habe & Liste wird angezeigt & \prio{Muss}\\
US-15 Mitglieder verwalten & Organisator & Mitglieder entfernen oder sperren & Regeln durchgesetzt werden & Zugriff wird entzogen & \prio{Muss}\\
US-16 Mindestteilnehmer festlegen & Organisator & eine Mindestzahl definieren & Event nur bei genug Teilnehmern stattfindet & Status wird automatisch geprüft & \prio{Sollte}\\
}

\ustable{FG4 Notizen und Umfragen}{
US-17 Notiz erstellen & Benutzer & eine Notiz anlegen & Infos dokumentiert sind & Notiz erscheint in der Liste & \prio{Muss}\\
US-18 Notiz bearbeiten & Benutzer & Notizen ändern & Fehler korrigiert werden & Änderungen werden gespeichert & \prio{Muss}\\
US-19 Notiz löschen & Benutzer & Notizen entfernen & irrelevante Infos verschwinden & Notiz wird entfernt & \prio{Muss}\\
US-20 Notizen filtern & Benutzer & nach Typ filtern & ich schneller finde, was ich suche & Filter zeigt passende Einträge & \prio{Sollte}\\
US-21 Termin erfassen & Benutzer & Termine mit Datum und Ort speichern & alle informiert sind & Termin enthält strukturierte Felder & \prio{Muss}\\
US-22 Umfrage erstellen & Benutzer & eine Umfrage anlegen & Entscheidungen getroffen werden & Antwortoptionen frei wählbar & \prio{Muss}\\
US-23 Abstimmen & Benutzer & an Umfragen teilnehmen & meine Meinung zählt & Stimme wird gezählt & \prio{Muss}\\
US-24 Ergebnis anzeigen & Benutzer & Ergebnisse sehen & Transparenz besteht & Stimmenverteilung wird angezeigt & \prio{Muss}\\
}

\ustable{FG5 Finanzen}{
US-25 Einzahlung tätigen & Benutzer & Geld einzahlen & Event finanziert wird & Betrag wird gespeichert & \prio{Muss}\\
US-26 Zahlungsstand anzeigen & Benutzer & Gesamtbetrag sehen & ich Überblick habe & eingegangenes Geld und Zielsumme werden angezeigt & \prio{Muss}\\
US-27 Ausgabe erfassen & Benutzer & Ausgaben dokumentieren & Kosten transparent sind & Betrag und Beschreibung werden gespeichert & \prio{Muss}\\
US-28 Kassenübersicht & Organisator & alle Einnahmen und Ausgaben sehen & ich abrechnen kann & Summen werden korrekt berechnet & \prio{Muss}\\
US-29 Rückzahlung auslösen & Organisator & Rückzahlungen durchführen & Guthaben ausgeglichen wird & Rückzahlung wird dokumentiert & \prio{Sollte}\\
}

\ustable{FG6 Aufgabenmanagement}{
US-30 Aufgabe erstellen & Organisator & Aufgaben anlegen & To-dos geplant sind & Aufgabe erscheint in Liste & \prio{Muss}\\
US-31 Aufgabe übernehmen & Benutzer & Aufgaben übernehmen & Zuständigkeit klar ist & Aufgabe wird mir zugeordnet & \prio{Muss}\\
US-32 Aufgabe abschließen & Benutzer & Aufgaben als erledigt markieren & Fortschritt sichtbar ist & Status ändert sich auf erledigt & \prio{Muss}\\
US-33 Erinnerung senden & System & an offene Aufgaben erinnern & nichts vergessen wird & Benachrichtigung wird automatisch gesendet & \prio{Sollte}\\
}

\ustable{FG7 Medien und Dokumentation}{
US-34 Medien hochladen & Benutzer & Fotos oder Videos hochladen & das Event dokumentiert ist & Dateien werden gespeichert und angezeigt & \prio{Sollte}\\
US-35 Kommentare schreiben & Benutzer & Bewertungen oder Kommentare hinterlassen & Erfahrungen geteilt werden & Text wird gespeichert & \prio{Könnte}\\
}

\ustable{FG8 Rollen- und Rechtesystem}{
US-36 Rollen vergeben & Institutions\-admin & Rollen zuweisen & Rechte gesteuert werden & Berechtigungen werden gespeichert & \prio{Muss}\\
US-37 Leserechte einschränken & Institutions\-admin & nur lesen können & Regeln eingehalten werden & Schreibfunktionen deaktiviert & \prio{Muss}\\
US-38 Lizenzlimit prüfen & Institutions\-admin & Events begrenzen & Community-Regeln gelten & mehr als drei Events werden verhindert & \prio{Sollte}\\
}

\ustable{FG9 Öffentliche Eventseite und QR-Code}{
US-39 Eventseite anzeigen & Organisator & eine öffentliche Seite haben & ich das Event bewerben kann & feste URL ist erreichbar & \prio{Sollte}\\
US-40 QR-Code generieren & Benutzer & einen QR-Code erhalten & ich den Link schnell teilen kann & Code wird korrekt erzeugt & \prio{Sollte}\\
}

\ustable{FG10 Integration und Benachrichtigungen}{
US-41 Kalender synchronisieren & Benutzer & Termine synchronisieren & sie automatisch im Kalender erscheinen & Synchronisation funktioniert & \prio{Könnte}\\
US-42 Benachrichtigungen senden & System & Nachrichten versenden & Nutzer informiert bleiben & Versand über gewählten Kanal erfolgt & \prio{Sollte}\\
US-42a Sprachassistent verbinden & Benutzer & einen digitalen Sprachassistenten koppeln & ich per Sprache auf meine Events zugreifen kann & Verbindung wird autorisiert und Konto ist verknüpft & \prio{Könnte}\\
US-42b Termine per Sprache abfragen & Benutzer & meine nächsten Termine per Sprachbefehl abrufen & ich Informationen freihändig erhalten kann & Assistent gibt korrekte Termine aus & \prio{Könnte}\\
US-42c Aktion per Sprache ausführen & Benutzer & einfache Aktionen wie Aufgaben abhaken oder Einzahlungen bestätigen per Sprache ausführen & ich schnell reagieren kann ohne die App zu öffnen & Aktion wird korrekt ausgeführt und im System gespeichert & \prio{Könnte}\\
}

\ustable{FG11 Dienstleister und Anfragen}{
US-43 Anfrage stellen & Kunde & ein Event anfragen & ich ein Angebot erhalte & Formular speichert alle Pflichtdaten & \prio{Muss}\\
US-44 Angebot erstellen & Dienstleister & ein Angebot generieren & ich Kunden informieren kann & Angebotsdokument wird erstellt & \prio{Muss}\\
US-45 Rechnung erstellen & Dienstleister & eine Rechnung erzeugen & ich abrechnen kann & Rechnung enthält alle Pflichtangaben & \prio{Muss}\\
US-46 Zahlung erinnern & System & an offene Rechnungen erinnern & Zahlungen rechtzeitig eingehen & Erinnerung wird automatisch versendet & \prio{Sollte}\\
}

\newpage

\ustable{FG12 Ressourcenplanung}{
US-47 Material verwalten & Dienstleister & Material anlegen oder ändern & Bestand aktuell bleibt & Materialdaten werden gespeichert & \prio{Muss}\\
US-48 Material buchen & Dienstleister & Material einem Event zuordnen & Doppelbuchungen vermieden werden & Material ist im Zeitraum gesperrt & \prio{Muss}\\
US-49 Mitarbeiter verwalten & Dienstleister & Mitarbeiter planen & Personal verfügbar ist & Einsatzplan berücksichtigt Verfügbarkeit & \prio{Muss}\\
US-50 Wartungshinweis & System & Wartungen anzeigen & Geräte sicher bleiben & Hinweis erscheint bei fälligem Zyklus & \prio{Sollte}\\
}

% =========================================================
\section{Nicht-funktionale Anforderungen (ISO 25010)}

\subsection{Performance Efficiency}
Die Plattform soll gemäß der Aufgabenstellung auch bei sehr vielen gleichzeitigen Benutzern eine hohe Leistungsfähigkeit aufweisen. Das System muss skalierbar sein und darf auch bei paralleler Nutzung mehrerer Gruppen und Events keine spürbaren Verzögerungen zeigen. Aktionen sollen innerhalb von maximal zwei Sekunden verarbeitet werden und der Seitenaufbau soll auch bei normaler Netzwerklatenz nur wenige Sekunden dauern. Das System soll mehrere tausende gleichzeitige Benutzer unterstützen und bei steigender Last einfach weiter skalierbar sein, z.\,B. durch weitere Serverinstanzen.

Zusätzlich werden zur Sicherstellung einer stabilen Performance sinnvolle ergänzende Maßnahmen definiert. Ressourcenintensive Prozesse wie Datei-Uploads oder größere Berechnungen sollen asynchron im Hintergrund ausgeführt werden, damit die Benutzeroberfläche weiterhin reaktionsfähig bleibt. Der Ressourcenverbrauch von CPU, Arbeitsspeicher und Datenbank soll überwacht und regelmäßig optimiert werden, um auch bei wachsendem Datenbestand eine gleichbleibende Leistung zu gewährleisten.


\subsection{Compatibility}
Das System soll auf unterschiedlichen Endgeräten und Softwareumgebungen nutzbar sein. Gemäß der Aufgabenstellung muss die Anwendung sowohl im Webbrowser als auch auf mobilen Endgeräten lauffähig sein und funktional konsistent sein. Alle wesentlichen Funktionen der Webversion sollen somit auch in der mobilen Anwendung in gleicher oder gleichwertiger Form zur Verfügung stehen.

Die Webanwendung soll in aktuellen Versionen der gängigen Browser Google Chrome, Microsoft Edge, Mozilla Firefox und Safari ohne zusätzliche Plugins oder Erweiterungen nutzbar sein. Für mobile Endgeräte soll eine Unterstützung der Betriebssysteme Android und iOS gewährleistet werden.

Darüber hinaus soll das System die Integration externer Dienste ermöglichen. Es sollen Kalenderdienste wie Google Calendar, Outlook und iCal angebunden werden können, sodass Termine automatisch synchronisiert werden. Weitere Schnittstellen, beispielsweise zu Zahlungs- oder Ticketingsystemen, sollen über REST-APIs realisiert werden.

Ergänzend werden folgende technische Maßnahmen geplant, um eine hohe Geräteunabhängigkeit sicherzustellen: Die Benutzeroberfläche soll responsiv gestaltet sein und sich automatisch an unterschiedliche Bildschirmgrößen und Auflösungen anpassen. Damit wird die explizit geforderte Endgeräteunabhängigkeit umgesetzt und durch zusätzliche Maßnahmen zur praktischen Nutzbarkeit abgesichert.

\subsection{Usability (Gebrauchstauglichkeit)}
Die Anwendung soll gemäß der Aufgabenstellung ohne lange Einarbeitung nutzbar sein. Neue Nutzer sollen grundlegende Funktionen wie das Erstellen oder Beitreten von Gruppen, das Anlegen von Notizen sowie das Verwalten von Aufgaben und Zahlungen intuitiv bedienen können, ohne zuvor eine Schulung oder umfangreiche Anleitung zu benötigen.

Die Benutzeroberfläche soll klar strukturiert und konsistent aufgebaut sein. Navigationselemente, Bezeichnungen und Strukturen sollen über alle Bereiche der Anwendung hinweg einheitlich verwendet werden, damit sich Nutzer schnell orientieren können. Zentrale Funktionen sollen mit wenigen Interaktionen erreichbar sein, sodass typische Aufgaben ohne nervige Zwischenschritte ausgeführt werden können.

Systemmeldungen und Fehlermeldungen sollen verständlich formuliert sein und dem Nutzer konkrete Hinweise zur Behebung von Problemen geben. Unterschiedliche Inhalte, wie beispielsweise Notiztypen oder Statusinformationen, sollen visuell eindeutig unterscheidbar dargestellt werden, etwa durch Farben oder Symbole.

Zusätzlich werden sinnvolle ergänzende Maßnahmen zur Verbesserung der Gebrauchstauglichkeit berücksichtigt. Die Oberfläche soll responsiv gestaltet sein, ausreichend große Bedienelemente besitzen und auch auf mobilen Geräten gut lesbar und bedienbar bleiben. Damit wird die explizit geforderte einfache Nutzbarkeit umgesetzt und durch weitere Maßnahmen zur praktischen Benutzerfreundlichkeit ergänzt.

\subsection{Reliability (Zuverlässigkeit)}
Das System soll zuverlässig betrieben werden können und die zentralen Funktionen der Plattform dauerhaft bereitstellen. Da über die Anwendung unter anderem organisatorische Informationen und Finanzdaten (Einzahlungen und Ausgaben) verwaltet werden, muss eine konsistente Speicherung dieser Daten gewährleistet sein, sodass keine Daten durch technische Störungen verloren gehen oder in einen widersprüchlichen Zustand geraten.

Zur Konkretisierung werden folgende Qualitätsziele festgelegt: Änderungen an relevanten Daten (z.\,B. Zahlungen, Ausgaben, Aufgabenstatus) sollen dauerhaft und konsistent gespeichert werden. Bei temporären Netzwerkproblemen oder kurzfristigen Serverstörungen sollen Benutzeraktionen nicht zu unvollständigen oder doppelten Einträgen führen. Außerdem sollen regelmäßig automatische Datensicherungen durchgeführt werden, sodass eine Wiederherstellung nach Systemausfällen möglich ist.

Ergänzend werden sinnvolle technische Maßnahmen definiert, um die Zuverlässigkeit praktisch abzusichern. Fehlerzustände sollen protokolliert werden (Logging), und kritische Systemzustände sollen überwacht werden (Monitoring), um Ausfälle frühzeitig zu erkennen. Für zentrale Prozesse sollen geeignete Mechanismen vorgesehen werden, die eine sichere Verarbeitung auch bei hoher Last unterstützen.

\subsection{Security (Sicherheit)}
Die Aufgabenstellung legt einen besonderen Fokus auf Sicherheitsanforderungen. Die Datenkommunikation zwischen Client und Server muss verschlüsselt erfolgen, damit Inhalte nicht von Unbefugten abgefangen werden können. Passwörter dürfen nicht im Klartext gespeichert werden und müssen in der Datenbank ausschließlich in gehashter und gesalzener Form vorliegen. Zusätzlich sollen Nutzer zu starken Passwörtern ermutigt werden und es soll eine Zwei-Faktor-Authentifizierung, beispielsweise per SMS oder vergleichbar, unterstützt werden.

Weiterhin ist aufgrund des Rollen- und Rechtesystems sicherzustellen, dass Benutzer nur auf diejenigen Gruppen- und Eventdaten zugreifen können, für die sie berechtigt sind. Insbesondere müssen administrative Aktionen wie das Entfernen oder Sperren von Mitgliedern auf Organisatoren bzw. Institutionsadministratoren beschränkt sein.

Ergänzend werden technische Schutzmaßnahmen festgelegt, die in modernen Web- und App-Systemen üblich sind. Dazu zählen der Schutz vor typischen Angriffen wie SQL-Injection, Cross-Site Scripting (XSS) und Cross-Site Request Forgery (CSRF) sowie sichere Sitzungsverwaltung (z.\,B. Timeout bei Inaktivität). Sicherheitsrelevante Ereignisse wie fehlgeschlagene Loginversuche oder Änderungen sicherheitskritischer Einstellungen sollen protokolliert werden.

\subsection{Maintainability (Wartbarkeit)}
Für eine langfristige Nutzung und Weiterentwicklung soll die Plattform wartbar und gut erweiterbar sein. Gemäß der Aufgabenstellung wird ein Client-Server-System umgesetzt, das auf mobilen Endgeräten und im Webbrowser verfügbar ist. Daraus ergibt sich eine klare Trennung zwischen Client, Server und Datenhaltung, sodass Änderungen an einzelnen Komponenten möglichst isoliert vorgenommen werden können.

Zur Konkretisierung werden folgende Qualitätsziele festgelegt: Schnittstellen zwischen den Systemkomponenten sollen klar definiert und dokumentiert sein. Insbesondere ist eine dokumentierte REST-API bereitzustellen, die die Kommunikation zwischen Client und Server nachvollziehbar beschreibt. Die Struktur des Systems soll so gestaltet sein, dass neue Funktionen (z.\,B. zusätzliche Integrationen oder neue Notiztypen) ohne grundlegende Umstrukturierung ergänzt werden können.

Ergänzend werden Maßnahmen vorgesehen, die Wartbarkeit praktisch unterstützen. Dazu gehören ein konsistenter Code-Stil, eine nachvollziehbare Struktur der Komponenten sowie die Möglichkeit, zentrale Funktionen automatisiert zu testen. Konfigurationswerte (z.\,B. Rollenrechte oder Systemparameter) sollen soweit sinnvoll ohne Codeänderungen anpassbar sein.

\subsection{Portability (Übertragbarkeit)}
Das System soll auf aktuellen Plattformen betrieben werden können und nicht von spezieller Hardware abhängig sein. Da die Anwendung sowohl im Webbrowser als auch auf mobilen Endgeräten genutzt wird, soll die Lösung auf den gängigen Betriebssystemen lauffähig sein. Dazu gehören insbesondere Android und iOS für mobile Geräte sowie aktuelle Desktop-Umgebungen für die Webnutzung.

Zur Konkretisierung wird festgelegt, dass die Serverkomponenten auf unterschiedlichen Serverumgebungen betrieben werden können sollen. Das Deployment soll standardisiert möglich sein, sodass ein Umzug zwischen unterschiedlichen Umgebungen (z.\,B. verschiedene Hosting-Anbieter) ohne große Anpassungen realisierbar bleibt.

Ergänzend werden sinnvolle technische Maßnahmen berücksichtigt, um die Übertragbarkeit zu erhöhen. Dazu zählen standardisierte Deploymentverfahren (z.\,B. Containerisierung) sowie der Export und Import von Daten in gängigen Formaten, sofern dies für administrative oder datenschutzbezogene Zwecke erforderlich ist.

\subsection{Datenschutz (nicht aus der ISO)}
Neben den ISO-Qualitätsmerkmalen sind in der Aufgabenstellung explizite Anforderungen an Datenschutz und Vertraulichkeit genannt. Nutzerdaten müssen gemäß Datenschutzgrundverordnung (DSGVO) behandelt werden, und die Datenschutzrichtlinie muss von den Nutzern bestätigt werden, bevor die Software vollständig genutzt werden kann. Private Informationen, insbesondere Kontaktinformationen und ausgetauschte Nachrichten, dürfen nicht in falsche Hände geraten. Die Aufgabenstellung fordert daher eine verschlüsselte Speicherung und einen verschlüsselten Austausch sensibler Daten; darüber hinaus wird ein Ende-zu-Ende-Verschlüsselungssystem als mögliche Lösung genannt.

Zur Konkretisierung werden folgende Anforderungen festgelegt: Es soll nur die für den Betrieb notwendige Menge personenbezogener Daten gespeichert werden (Datenminimierung). Nutzer sollen ihre eigenen Daten einsehen können und es soll ein Verfahren vorgesehen werden, um personenbezogene Daten auf Anfrage zu löschen. Der Zugriff auf personenbezogene Daten muss durch geeignete Berechtigungen geschützt werden.

Ergänzend werden sinnvolle Maßnahmen vorgesehen, um Datenschutzanforderungen praktisch umzusetzen. Dazu gehören ein nachvollziehbares Lösch- und Aufbewahrungskonzept für Daten, sowie eine sichere Protokollierung, ohne dabei vertrauliche Inhalte im Klartext zu speichern.

% =========================================================
\section{Graphische Benutzungsschnittstelle}

\subsection{Designprinzipien und UX-Grundlagen}

\subsubsection{Zielsetzung der Benutzeroberfläche}
Die Benutzeroberfläche von EventHub soll eine einfache und intuitive Bedienung der Plattform ermöglichen. Nutzer sollen typische Aufgaben wie Gruppen beitreten, Events verwalten, Aufgaben bearbeiten oder Zahlungen durchführen ohne Schulung ausführen können. Die Oberfläche ist daher klar strukturiert, reduziert gestaltet und auf die wesentlichen Funktionen fokussiert.

\subsubsection{Konsistenz und Navigationskonzept}
Alle Bereiche der Anwendung verwenden ein einheitliches Layout und wiederkehrende Navigationselemente. Gleichartige Funktionen sind an denselben Positionen erreichbar, sodass sich Nutzer schnell orientieren können. Zentrale Aktionen sind direkt sichtbar und nicht in tiefen Menüstrukturen versteckt. Häufig genutzte Funktionen sind mit maximal zwei bis drei Interaktionen erreichbar.

\subsubsection{Rollenbasierte Darstellung}
Die Oberfläche ist rollenabhängig gestaltet. Benutzer sehen ausschließlich Funktionen, für die sie berechtigt sind. Teilnehmer erhalten eine vereinfachte Ansicht mit Fokus auf Teilnahme und Interaktion, während Organisatoren und Institutionsadministratoren zusätzliche Verwaltungsfunktionen erhalten. Dadurch wird die Komplexität reduziert und Fehlbedienung vermieden.

\subsubsection{Responsives Design und Geräteunabhängigkeit}
Die Anwendung ist responsiv gestaltet und passt sich automatisch an unterschiedliche Bildschirmgrößen und Endgeräte an. Die Webversion ist für Desktop-Browser optimiert, während mobile Endgeräte eine vereinfachte Navigation und größere Bedienelemente erhalten. Alle wesentlichen Funktionen stehen auf beiden Plattformen in gleichwertiger Form zur Verfügung.

\subsubsection{Fehlervermeidung und Systemfeedback}
Pflichtfelder werden eindeutig gekennzeichnet und Eingaben vor dem Speichern validiert. Kritische Aktionen wie Löschen oder Rückzahlungen werden durch Bestätigungsdialoge abgesichert. Das System informiert Nutzer jederzeit über den aktuellen Status, beispielsweise durch Ladeanzeigen, Erfolgsmeldungen oder verständliche Fehlermeldungen. Dadurch bleibt das Systemverhalten nachvollziehbar und transparent.

\subsubsection{Barrierearme Bedienbarkeit}
Texte sind gut lesbar gestaltet und Bedienelemente ausreichend groß dimensioniert, sodass die Anwendung auch auf mobilen Geräten oder Touchscreens komfortabel nutzbar ist. Kontraste und klare Beschriftungen unterstützen eine schnelle Erfassbarkeit der Inhalte.

% ---------------------------------------------------------
\subsection{Navigationsmodelle}
Die folgenden Zustandsdiagramme zeigen Navigation und mögliche Zustandsübergänge der Benutzeroberfläche.

\subsubsection{Zustandsdiagramm -- Privatnutzer}
\begin{figure}[H]
  \centering
  \includegraphics[width=\linewidth]{pngs/zustandsdiagramm_privatnutzer.png}
  \caption{Zustandsdiagramm Privatnutzer}
\end{figure}

\subsubsection{Zustandsdiagramm -- Organisator}
\begin{figure}[H]
  \centering
  \includegraphics[width=\linewidth]{pngs/zustandsdiagramm_organisator.png}
  \caption{Zustandsdiagramm Organisator}
\end{figure}

\subsubsection{Zustandsdiagramm -- Dienstleister}
\begin{figure}[H]
  \centering
  \includegraphics[width=\linewidth]{pngs/zustandsdiagramm_dienstleister.png}
  \caption{Zustandsdiagramm Dienstleister}
\end{figure}

% ---------------------------------------------------------
\subsection{Webbrowser-Mockups}

\subsubsection{Login \& Registrierung}
\begin{figure}[H]
  \centering
  \includegraphics[width=\linewidth]{mockups/login.png}
  \caption{UI-Mockup Login und Registrierung (US-01, US-02, US-03, US-04, US-05)}
\end{figure}

\subsubsection{Dashboard}
\begin{figure}[H]
  \centering
  \includegraphics[width=\linewidth]{mockups/Dashboard.png}
  \caption{UI-Mockup Dashboard (US-10, US-12, US-30, US-42)}
\end{figure}

\subsubsection{Gruppenübersicht}
\begin{figure}[H]
  \centering
  \includegraphics[width=\linewidth]{mockups/Gruppenuebersicht.png}
  \caption{UI-Mockup Gruppenübersicht (US-10, US-11, US-12, US-13, US-14)}
\end{figure}

\subsubsection{Event-Detailseite}
\begin{figure}[H]
  \centering
  \includegraphics[width=\linewidth]{mockups/Event-Detail.png}
  \caption{UI-Mockup Event-Detailseite (US-14, US-16, US-17--US-24, US-25--US-32, US-34, US-35, US-39, US-40)}
\end{figure}

\subsubsection{Finanzen}
\begin{figure}[H]
  \centering
  \includegraphics[width=\linewidth]{mockups/Finanzen.png}
  \caption{UI-Mockup Finanzen (US-25, US-26, US-27, US-28, US-29)}
\end{figure}

\subsubsection{Notizen \& Umfragen}
\begin{figure}[H]
  \centering
  \includegraphics[width=\linewidth]{mockups/Notizen_Umfragen.png}
  \caption{UI-Mockup Notizen und Umfragen (US-17, US-18, US-19, US-20, US-21, US-22, US-23, US-24)}
\end{figure}

\subsubsection{Aufgabenplanung}
\begin{figure}[H]
  \centering
  \includegraphics[width=\linewidth]{mockups/Aufgabenübersicht.png}
  \caption{UI-Mockup Aufgabenübersicht (US-30, US-31, US-32, US-33)}
\end{figure}

\subsubsection{Profil \& Einstellungen}
\begin{figure}[H]
  \centering
  \includegraphics[width=\linewidth]{mockups/Profil.png}
  \caption{UI-Mockup Profil \& Einstellungen (US-03, US-06, US-07, US-08, US-09, US-41, US-42a, US-42b, US-42c)}
\end{figure}

\subsubsection{Dienstleister-Dashboard}
\begin{figure}[H]
  \centering
  \includegraphics[width=\linewidth]{mockups/dienstleister_dashboard.png}
  \caption{UI-Mockup Dienstleister-Dashboard (US-43, US-44, US-45, US-46, US-47, US-48, US-49)}
\end{figure}

\subsubsection{Rollen- und Rechteverwaltung}
\begin{figure}[H]
  \centering
  \includegraphics[width=\linewidth]{mockups/rollen_rechteverwaltung.png}
  \caption{UI-Mockup Rollen- und Rechteverwaltung (US-36, US-37)}
\end{figure}

\subsubsection{Eventanfrage}
\begin{figure}[H]
  \centering
  \includegraphics[width=\linewidth]{mockups/eventanfrage.png}
  \caption{UI-Mockup Eventanfrage (US-43)}
\end{figure}

\subsubsection{Eventübersicht (Lesemodus)}
\begin{figure}[H]
  \centering
  \includegraphics[width=\linewidth]{mockups/eventuebersicht_lesemodus.png}
  \caption{UI-Mockup Eventübersicht (Lesemodus) (US-37)}
\end{figure}

% ---------------------------------------------------------
\subsection{Mobile-App-Mockups}

\subsubsection{Dashboard (Mobil)}
\begin{figure}[H]
  \centering
  \includegraphics[width=0.6\linewidth]{mockups/mobile_dashboard.png}
  \caption{UI-Mockup Dashboard Mobil (US-10, US-12, US-30, US-42)}
\end{figure}

\subsubsection{Event-Details (Mobil)}
\begin{figure}[H]
  \centering
  \includegraphics[width=0.6\linewidth]{mockups/mobile_eventdetail.png}
  \caption{UI-Mockup Event-Details Mobil (US-14, US-17, US-18, US-19, US-21, US-22, US-23, US-24, US-25, US-26, US-30, US-31, US-32)}
\end{figure}

\subsubsection{Aufgaben (Mobil)}
\begin{figure}[H]
  \centering
  \includegraphics[width=0.6\linewidth]{mockups/mobile_aufgaben.png}
  \caption{UI-Mockup Aufgaben Mobil (US-30, US-31, US-32, US-33)}
\end{figure}

\subsubsection{Finanzen (Mobil)}
\begin{figure}[H]
  \centering
  \includegraphics[width=0.6\linewidth]{mockups/mobile_finanzen.png}
  \caption{UI-Mockup Finanzen Mobil (US-25, US-26, US-27, US-28, US-29)}
\end{figure}

% =========================================================





\section{Logische Sicht}

\subsection{Ableitung der Domänenklassen aus den User Stories}

Die Domänenklassen des Systems EventHub wurden systematisch aus den fachlichen
Anforderungen und User Stories abgeleitet. Ziel war es, zentrale Konzepte des
Anwendungsbereichs als eigenständige Klassen zu identifizieren und ihre Beziehungen
zueinander fachlich korrekt abzubilden.

Aus den User Stories zur Benutzer- und Gruppenverwaltung (z.\,B. Registrierung,
Gruppenerstellung, Gruppenbeitritt) ergeben sich die Klassen \textit{Benutzer},
\textit{Gruppe}, \textit{Mitgliedschaft} und \textit{Rolle}. Die Mitgliedschaft
modelliert dabei explizit die Beziehung zwischen Benutzern und Gruppen inklusive
Status und Rolle.

User Stories zur Eventplanung und -durchführung führen zur Klasse \textit{Event}
sowie zu abhängigen Konzepten wie \textit{Aufgabe}, \textit{AufgabenZuweisung},
\textit{Medium} und \textit{Kommentar}. Diese Klassen ermöglichen die strukturierte
Organisation und Dokumentation von Events.

Funktionen zur Kommunikation und Abstimmung innerhalb von Gruppen werden durch
Notizen und Umfragen abgebildet. Daraus ergeben sich die Klassen \textit{Notiz}
mit ihren Spezialisierungen (Freitext-, Termin- und Ausgaben-Notiz) sowie
\textit{Umfrage}, \textit{UmfrageOption} und \textit{Stimme}.

Die User Stories zur Finanzverwaltung (Einzahlungen, Rückzahlungen, Kassenübersicht)
resultieren in den Klassen \textit{Einzahlung}, \textit{Rueckzahlung} und
berichtenden Strukturen wie dem \textit{Kassensturz}.

Erweiterte User Stories für Dienstleisterprozesse und Ressourcenplanung führen zu
Domänenklassen wie \textit{Dienstleister}, \textit{Anfrage}, \textit{Angebot},
\textit{Rechnung}, \textit{Mahnung} sowie \textit{Material}, \textit{Mitarbeiter}
und den zugehörigen Buchungs- und Einsatzklassen.

Die so identifizierten Domänenklassen bilden die fachliche Grundlage für das
Klassendiagramm und dienen als Ausgangspunkt für das logische Datenmodell.

\subsection{Fachliches Domänenmodell (UML-Klassendiagramm)}

\begin{figure}[H]
    \centering
    \includegraphics[width=1.0\textwidth]{Klassendiagramm1.png}
    \caption{Klassendiagramm1.png}
    \label{fig:klassendiagramm}
\end{figure}

\subsection{Beziehungen, Multiplizitäten und Vererbungen}

\subsection{Logisches Datenmodell}

\subsubsection{Benutzer, Gruppen und Events}

\begin{figure}[H]
    \centering
    \includegraphics[width=1.0\textwidth]{ldm_user_group_event.png}
    \caption{LDM Benutzer, Gruppen und Events}
    \label{fig:ldm1}
\end{figure}

Das logische Datenmodell bildet die zentralen Stammdaten des Systems EventHub ab.
Die Tabellen \texttt{users}, \texttt{groups} und \texttt{events} repräsentieren
Benutzer, Gruppen sowie geplante oder durchgeführte Events und stellen damit
die fachliche Grundlage der Anwendung dar.

Die Beziehung zwischen Benutzern und Gruppen wird über die Tabelle
\texttt{group\_memberships} modelliert. Dadurch können zusätzliche Informationen
wie Rolle und Status einer Mitgliedschaft (z.\,B. aktiv, gesperrt oder verlassen)
explizit gespeichert werden. Rollen werden separat verwaltet, um eine flexible
Zuordnung von Rechten innerhalb einer Gruppe zu ermöglichen.

Institutionen werden über die Tabelle \texttt{institutions} abgebildet. Die
Mitgliedschaft von Benutzern in einer Institution erfolgt über
\texttt{institution\_members}. Gruppen können optional einer Institution
zugeordnet sein, wodurch sowohl institutionelle als auch private Gruppen
unterstützt werden.

Events sind eindeutig einer Gruppe zugeordnet und enthalten neben zeitlichen
Angaben auch fachliche Attribute wie Mindestteilnehmerzahl, Mindestbudget
und Status. Eindeutige Felder wie E-Mail-Adresse, Benutzername, Beitrittscode
und öffentliche Event-URLs werden durch \texttt{unique}-Constraints abgesichert.
Alle Beziehungen sind über Fremdschlüssel eindeutig definiert.

\subsubsection{Notizen \& Umfragen}

\begin{figure}[H]
    \centering
    \includegraphics[width=1.0\textwidth]{ldm_notes_polls.png}
    \caption{LDM Notizen \& Umfragen}
    \label{fig:ldm2}
\end{figure}

Abbildung~\ref{fig:ldm2} zeigt das logische Datenmodell für gruppenbezogene
Notizen und Umfragen. Die Tabelle \texttt{notes} speichert alle Notizen als gemeinsame
Basisstruktur und unterscheidet über \texttt{note\_type} zwischen Freitext-, Termin-
und Ausgaben-Notizen.

Termin- und Ausgaben-spezifische Attribute werden in den Tabellen
\texttt{note\_appointments} bzw. \texttt{note\_expenses} abgelegt. Diese Tabellen
verwenden \texttt{note\_id} gleichzeitig als Primärschlüssel und Fremdschlüssel auf
\texttt{notes}, wodurch eine 1:1-Beziehung zwischen Basisnotiz und Spezialisierung
erzielt wird.

Umfragen werden über \texttt{polls} modelliert und sind eindeutig einer Gruppe sowie
einem Ersteller zugeordnet. Die Antwortmöglichkeiten sind in \texttt{poll\_options}
abgebildet; abgegebene Stimmen werden als Bewegungsdaten in \texttt{poll\_votes}
gespeichert. Dadurch sind Auswertungen (z.\,B. Ergebnisdarstellung) möglich, ohne
Umfragedaten nachträglich zu verändern.

\subsubsection{Finanzen \& Aufgaben}

\begin{figure}[H]
    \centering
    \includegraphics[width=1.0\textwidth]{ldm_payments_tasks.png}
    \caption{LDM Finanzen \& Aufgaben}
    \label{fig:ldm3}
\end{figure}

Das logische Datenmodell für Finanzen und Aufgaben bildet die zentralen
Bewegungsdaten im Kontext von Gruppen und Events ab. Finanzielle Transaktionen
werden über die Tabellen \texttt{payments} und \texttt{refunds} modelliert und
ermöglichen die Nachverfolgung von Einzahlungen sowie Rückzahlungen innerhalb
einer Gruppe.

Einzahlungen (\texttt{payments}) sind jeweils einem Benutzer und einer Gruppe
zugeordnet und enthalten neben dem Betrag auch Informationen zum Zahlungszeitpunkt,
Zahlungskanal und Status. Rückzahlungen (\texttt{refunds}) referenzieren ebenfalls
eine Gruppe und einen Benutzer als Empfänger; zusätzlich wird der auslösende
Benutzer (z.\,B. ein Organisator) separat gespeichert, um Verantwortlichkeiten
nachvollziehbar abzubilden.

Aufgaben werden über die Tabelle \texttt{tasks} modelliert und sind eindeutig
einem Event sowie einem Ersteller zugeordnet. Die konkrete Zuweisung von Aufgaben
an Benutzer erfolgt über \texttt{task\_assignments}. Dadurch können Aufgaben
mehreren Benutzern zugewiesen und Zustände wie Übernahme oder Erinnerung
zeitlich dokumentiert werden.

Durch die Trennung von Stammdaten (Aufgaben) und Bewegungsdaten (Zuweisungen,
Zahlungen, Rückzahlungen) bleibt das Datenmodell flexibel, erweiterbar und
konsistent.

\subsubsection{Dienstleister \& Abrechnung}

\begin{figure}[H]
    \centering
    \includegraphics[width=1.3\textwidth]{ldm_providers_invoices.png}
    \caption{LDM Dienstleister \& Abrechnung}
    \label{fig:ldm4}
\end{figure}

Das logische Datenmodell für Dienstleister und Abrechnung bildet den vollständigen
B2B-Prozess von der Anfrage über Angebot und Rechnung bis hin zu Mahnungen ab.
Dienstleister werden in der Tabelle \texttt{providers} verwaltet und können optional
mit einem Benutzerkonto verknüpft sein, sofern ein eigener Dienstleister-Login
existiert.

Anfragen an Dienstleister werden über \texttt{service\_requests} modelliert und
enthalten organisatorische sowie zeitliche Rahmenbedingungen eines geplanten Events.
Auf Basis einer Anfrage kann ein oder mehrere Angebote (\texttt{offers}) entstehen,
die Preisangaben und einen Angebotsstatus enthalten.

Wird ein Angebot angenommen, kann daraus eine Rechnung (\texttt{invoices}) erzeugt
werden. Rechnungen enthalten eindeutige Rechnungsnummern, Beträge, Zahlungsfristen
sowie Statusinformationen. Mahnungen werden in \texttt{reminders} separat gespeichert
und referenzieren jeweils eine Rechnung, wodurch mehrere Mahnstufen abgebildet werden
können.

Bewertungen von Dienstleistern erfolgen über \texttt{provider\_ratings} und sind
einem Benutzer sowie einem Dienstleister zugeordnet. Durch die klare Trennung der
einzelnen Prozessschritte bleibt das Datenmodell nachvollziehbar, erweiterbar und
fachlich konsistent.

\subsubsection{Ressourcenplanung}

\begin{figure}[H]
    \centering
    \includegraphics[width=1.0\textwidth]{ldm_resource_planning.png}
    \caption{LDM Ressourcenplanung}
    \label{fig:ldm5}
\end{figure}

Das logische Datenmodell zur Ressourcenplanung bildet Material- und Personaleinsatz
im Kontext von Dienstleisterangeboten ab. Materialien werden in \texttt{materials}
verwaltet und sind eindeutig einem Dienstleister (\texttt{provider\_id}) zugeordnet.
Über \texttt{material\_categories} können Materialien thematisch klassifiziert werden,
wobei Kategorienamen durch einen \texttt{unique}-Constraint eindeutig sind.

Konkrete Materialreservierungen für ein Angebot werden über \texttt{material\_bookings}
modelliert. Jede Buchung referenziert ein \texttt{offer} sowie ein \texttt{material} und
enthält Zeitraum und Menge, wodurch parallele Buchungen und wiederholte Ausleihen
abbildbar sind. Wartungsereignisse werden in \texttt{material\_maintenance} als eigene
Bewegungsdaten gespeichert und referenzieren das jeweilige Material, um Fälligkeiten
und durchgeführte Wartungen nachvollziehbar zu dokumentieren.

Mitarbeiter werden in \texttt{employees} verwaltet und ebenfalls einem Dienstleister
zugeordnet. Die Einsatzplanung erfolgt über \texttt{employee\_assignments}, die einen
Mitarbeiter mit einem Angebot verknüpft und Zeitraum sowie Einsatzstatus
(z.\,B. geplant, bestätigt, krank, Urlaub) speichert. Insgesamt ermöglicht das Modell
eine konsistente Planung und Nachverfolgung von Ressourcen pro Angebot.

\subsection{Übersicht der Teilmodelle und Querverweise}

\begin{figure}[H]
    \centering
    \includegraphics[width=1.0\textwidth]{ldm_connection.png}
    \caption{LDM Zusammenhang der LDM's}
    \label{fig:ldm6}
\end{figure}

Abbildung~\ref{fig:ldm6} zeigt eine kompakte Übersicht über den Zusammenhang
der fünf logischen Teilmodelle. Zentrale Entitäten sind \texttt{users}, \texttt{groups}
und \texttt{events}, auf denen die Community-Funktionen wie Notizen, Umfragen, Finanzen
und Aufgaben aufbauen.

Notizen und Umfragen sind gruppenbezogen modelliert, ebenso Einzahlungen und
Rückzahlungen. Aufgaben sind an Events gekoppelt und werden damit indirekt über die
zugehörige Gruppe kontextualisiert. Der Dienstleisterprozess verläuft von
\texttt{service\_requests} über \texttt{offers} zu \texttt{invoices} und ist über
\texttt{providers} mit Dienstleistern verknüpft.

Die Ressourcenplanung (Material und Mitarbeiter) ist dem Dienstleister zugeordnet und
wird über Angebote konkret für einzelne Aufträge/Events gebucht bzw. eingeplant.
Die Übersicht dient als Navigationshilfe und stellt sicher, dass die Teilmodelle
konsistent aufeinander referenzieren.

\newpage

\subsection{CRUD-Matrix}

{\small
\setlength{\LTpre}{0pt}
\setlength{\LTpost}{0pt}

\begin{longtable}{ |p{3cm}|p{2.5cm}|p{2.5cm}|p{2.5cm}|p{2.5cm}| }
\caption{CRUD-Matrix}\\
\hline
\textbf{Entität (Tabelle)} & \textbf{Create (C)} & \textbf{Read (R)} & \textbf{Update (U)} & \textbf{Delete (D)} \\
\hline
\endfirsthead

\hline
\textbf{Entität (Tabelle)} & \textbf{Create (C)} & \textbf{Read (R)} & \textbf{Update (U)} & \textbf{Delete (D)} \\
\hline
\endhead

% --- Teilmodell 1: Benutzer/Gruppe/Event ---
users &
US-01 &
US-02, US-06 &
US-04, US-05, US-07, US-08, US-09 &
-- \\
\hline

institutions &
-- &
US-38 &
-- &
-- \\
\hline

institution\_members &
-- &
-- &
US-36, US-37 &
-- \\
\hline

groups &
US-10 &
US-14 &
US-15 (Code verwalten implizit), US-16 (Min. Teilnehmer/Budget) &
US-11 \\
\hline

group\_memberships &
US-10 (Organisator wird Mitglied), US-12 &
US-14 &
US-13 (Austritt als Statuswechsel), US-15 (sperren/entfernen), US-36, US-37 &
-- \\
\hline

events &
US-10 (Event im Rahmen der Gruppe anlegen), US-39 (öffentliche Seite implizit) &
US-39 &
US-16 &
US-11 (archivieren/löschen) \\
\hline

% --- Teilmodell 2: Notizen/Umfragen ---
notes &
US-17, US-21, US-27 &
US-20 &
US-18 &
US-19 \\
\hline

note\_appointments &
US-21 &
US-21, US-41 &
US-18 (wenn Termin-Notiz bearbeitet wird) &
US-19 (via Notiz löschen) \\
\hline

note\_expenses &
US-27 &
US-26, US-28 &
US-18 (wenn Ausgabe-Notiz bearbeitet wird) &
US-19 (via Notiz löschen) \\
\hline

polls &
US-22 &
US-24 &
-- &
-- \\
\hline

poll\_options &
US-22 &
US-24 &
-- &
-- \\
\hline

poll\_votes &
US-23 &
US-24 &
-- &
-- \\
\hline

% --- Teilmodell 3: Finanzen/Aufgaben ---
payments &
US-25 &
US-26, US-28 &
-- &
-- \\
\hline

refunds &
US-29 &
US-28 &
-- &
-- \\
\hline

tasks &
US-30 &
-- &
US-32 &
-- \\
\hline

task\_assignments &
US-31 &
-- &
US-33 (Erinnerung/Statusdaten) &
-- \\
\hline

% --- Medien/Kommentare (falls in eurem Modell) ---
media &
US-34 &
US-34 &
-- &
-- \\
\hline

comments &
US-35 &
-- &
-- &
-- \\
\hline

% --- Teilmodell 4: Dienstleister/Abrechnung ---
providers &
-- &
-- &
-- &
-- \\
\hline

service\_requests &
US-43 &
-- &
-- &
-- \\
\hline

offers &
US-44 &
-- &
-- &
-- \\
\hline

invoices &
US-45 &
-- &
-- &
-- \\
\hline

reminders &
US-46 &
-- &
-- &
-- \\
\hline

provider\_ratings &
-- &
-- &
-- &
-- \\
\hline

% --- Teilmodell 5: Ressourcenplanung ---
material\_categories &
-- &
-- &
-- &
-- \\
\hline

materials &
US-47 &
US-47, US-50 &
US-47 &
US-47 (falls ``entfernen'' vorgesehen) \\
\hline

material\_maintenance &
-- &
US-50 &
-- &
-- \\
\hline

material\_bookings &
US-48 &
-- &
-- &
-- \\
\hline

employees &
US-49 &
US-49 &
US-49 &
US-49 (falls ``entfernen'' vorgesehen) \\
\hline

employee\_assignments &
US-49 (Einsatzplanung implizit) &
US-49 &
US-49 &
-- \\
\hline

\end{longtable}
\noindent
\\
Die CRUD-Matrix zeigt die Zuordnung der User Stories zu den persistierten Entitäten.
Bewegungsdaten (z.\,B. Zahlungen, Abstimmungen, Buchungen) werden ausschließlich erzeugt
(Create) und gelesen (Read), jedoch nicht verändert oder gelöscht.


\section{Prozess-Sicht}

\subsection{Zustandsdiagramm: Event}

\begin{figure}[H]
    \centering
    \includegraphics[width=1.0\textwidth]{Zustandsdiagramm_Event.png}
    \caption{Zustandsdiagramm\_Event.png}
    \label{fig:zustandsdiagramm_event}
\end{figure}

Das Zustandsdiagramm beschreibt den Lebenszyklus eines Events von der initialen
Erstellung bis zur Archivierung oder Löschung. Nach dem Anlegen befindet sich ein
Event zunächst im Zustand \textit{Entwurf} und kann geplant, geändert oder verworfen
werden.

Im Zustand \textit{Geplant} werden fachliche Bedingungen wie Mindestteilnehmerzahl
und Mindestbudget geprüft. Sind diese erfüllt, wird das Event freigeschaltet und kann
zum geplanten Zeitpunkt oder manuell in den Zustand \textit{Aktiv} übergehen. Nach
Abschluss der Veranstaltung wechselt das Event in den Zustand \textit{Abgeschlossen}
und kann anschließend archiviert oder gemäß definierter Regeln gelöscht werden.

Das Diagramm stellt sicher, dass alle fachlich zulässigen Zustandsübergänge eindeutig
definiert sind und der Eventstatus jederzeit konsistent ist.

\subsection{Zustandsdiagramm: Rechnung}

\begin{figure}[H]
    \centering
    \includegraphics[width=1.0\textwidth]{Zustandsdiagramm_Rechnung.png}
    \caption{Zustandsdiagramm Rechnung}
    \label{fig:zustandsdiagramm_rechnung}
\end{figure}

Das Zustandsdiagramm für Rechnungen modelliert den fachlichen Ablauf von der Erstellung
einer Rechnung bis zu deren Abschluss. Eine Rechnung wird zunächst im Zustand
\textit{offen} angelegt und kann durch Zahlung in den Zustand \textit{bezahlt}
übergehen.

Erfolgt innerhalb der definierten Frist keine Zahlung, wechselt die Rechnung in den
Zustand \textit{überfällig}. In diesem Zustand können Mahnungen erzeugt werden, wodurch
der Status entsprechend aktualisiert wird. Nach vollständigem Zahlungseingang gilt
die Rechnung als abgeschlossen.

Durch die Modellierung der Zustände wird sichergestellt, dass Rechnungen nachvollziehbar
bearbeitet werden und finanzielle Prozesse transparent und regelkonform ablaufen.

\subsection{Sequenzdiagramm: Gruppe per Beitrittscode beitreten}

\begin{figure}[H]
    \centering
    \includegraphics[width=1.0\textwidth]{Sequenzdiagramm_Gruppebeitreten.png}
    \caption{Sequenzdiagramm Gruppe per Beitritscode beitreten}
    \label{fig:seq-join-group}
\end{figure}

Abbildung~\ref{fig:seq-join-group} zeigt den Ablauf eines Gruppenbeitritts über einen
Beitrittscode. Der Benutzer initiiert den Vorgang über den Web- oder Mobile-Client,
welcher einen REST-Request an die Server-Schnittstelle sendet.

Die Server-Schicht prüft die Gültigkeit des Beitrittscodes, den Status der Gruppe
sowie bestehende Mitgliedschaften. Ist der Beitritt zulässig, wird eine neue
Mitgliedschaft mit der Rolle \textit{Mitglied} angelegt und persistiert. Andernfalls
wird eine geeignete Fehlermeldung an den Client zurückgegeben.

Das Diagramm verdeutlicht die Trennung von Präsentation, Geschäftslogik und
Datenhaltung sowie die serverseitige Durchsetzung fachlicher Regeln.

\subsection{Sequenzdiagramm: Einzahlung erfassen}

\begin{figure}[H]
    \centering
    \includegraphics[width=1.0\textwidth]{Sequenzdiagramm_Einzahlungerfassen.png}
    \caption{Sequenzdiagramm Einzahlung erfassen}
    \label{fig:seq-payment}
\end{figure}

Abbildung~\ref{fig:seq-payment} zeigt den Ablauf zur Erfassung einer Einzahlung durch
ein Gruppenmitglied. Der Benutzer gibt den Betrag im Client ein, woraufhin ein
authentifizierter REST-Request an die Server-Schnittstelle gesendet wird.

Die Server-Schicht prüft zunächst, ob der Benutzer aktives Mitglied der angegebenen
Gruppe ist. Ist dies der Fall, wird die Einzahlung als Bewegungsdatum in der Datenbank
persistiert und dem Client eine erfolgreiche Bestätigung zurückgegeben. Andernfalls
wird der Vorgang mit einem Fehlerstatus abgelehnt.

Das Sequenzdiagramm verdeutlicht die serverseitige Durchsetzung fachlicher Regeln
sowie die klare Trennung zwischen Client, Geschäftslogik und Datenhaltung.

\section{Entwicklungs-Sicht}

\subsection{Drei-Schichten-Architektur (Client – Server – Datenbank)}

EventHub wird als klassische Drei-Schichten-Architektur umgesetzt. Ziel ist eine klare
Trennung von Verantwortlichkeiten zwischen Darstellung, Geschäftslogik und Datenhaltung.
Dadurch werden Wartbarkeit, Skalierbarkeit und Sicherheit verbessert.

\subsection{Client-Schicht (Präsentation)}

Die Client-Schicht umfasst die Webanwendung (Browser) sowie die Mobile App. Sie ist
verantwortlich für:
\begin{itemize}
  \item Benutzerinteraktion (Formulare, Navigation, Validierung auf UI-Ebene)
  \item Darstellung von Gruppen, Events, Notizen, Umfragen, Finanzen und Aufgaben
  \item Aufruf der Backend-Schnittstellen über HTTPS (REST)
  \item Token-basierte Authentifizierung (z.\,B. Speicherung eines Access Tokens)
\end{itemize}

\noindent
Die Logik (z.\,B. Rollenprüfung oder Finanzberechnungen) liegt nicht im
Client, sondern in der Server-Schicht, um eine Umgehung fachlicher Regeln
zu verhindern.

\subsection{Server-Schicht (Applikation / Geschäftslogik)}

Die Server-Schicht stellt die zentrale REST-API bereit und enthält die vollständige
Geschäftslogik der Anwendung. Sie übernimmt unter anderem:
\begin{itemize}
  \item Authentifizierung (Login, Zwei-Faktor-Authentifizierung) und Autorisierung
        (Rollen und Rechte wie Organisator, Mitglied, Institutionsadministrator,
        Dienstleister)
  \item Validierung fachlicher Regeln (z.\,B. Beitrittscode, Mindestteilnehmer,
        Mindestbudget, Sperrstatus)
  \item Orchestrierung komplexer Use Cases (z.\,B. Anfrage $\rightarrow$ Angebot
        $\rightarrow$ Rechnung $\rightarrow$ Mahnung)
  \item Aggregationen und Berichte (z.\,B. Kassenübersicht aus Einzahlungen, Ausgaben
        und Rückzahlungen)
  \item Integration externer Dienste (z.\,B. Kalendersynchronisation oder
        Benachrichtigungskanäle) über Adapter oder Services
\end{itemize}

Die Server-Schicht kapselt sämtliche Datenzugriffe (z.\,B. über Repository- oder
DAO-Konzepte) und gibt nach außen ausschließlich kontrollierte Datenobjekte
(DTOs/JSON) zurück.

\subsection{Datenbank-Schicht (Persistenz)}

Die Datenbank-Schicht ist für die persistente Speicherung aller relevanten Daten
verantwortlich. Dazu zählen unter anderem:
\begin{itemize}
  \item Stammdaten: Benutzer, Gruppen, Events, Dienstleister, Material, Mitarbeiter
  \item Bewegungsdaten: Einzahlungen, Abstimmungen, Buchungen, Einsätze, Mahnungen
  \item Status- und Historieninformationen (z.\,B. Mitgliedschaft aktiv, gesperrt oder
        verlassen)
\end{itemize}

Zugriffe auf die Datenbank erfolgen ausschließlich über die Server-Schicht. Ein direkter
Datenbankzugriff vom Client ist nicht möglich. Dadurch können Rechteprüfungen,
Datenschutzmaßnahmen und Konsistenzregeln zentral durchgesetzt werden.

\subsection{Datenfluss}

Der typische Datenfluss innerhalb der Drei-Schichten-Architektur gestaltet sich wie
folgt:
\begin{enumerate}
  \item Der Client sendet einen Request (HTTPS/REST) an den Server, inklusive
        Authentifizierungs-Token.
  \item Der Server prüft Token und Berechtigungen und führt die entsprechende
        Geschäftslogik aus.
  \item Der Server liest oder schreibt Daten in der Datenbank.
  \item Der Server antwortet mit einem JSON-Ergebnis, das vom Client dargestellt wird.
\end{enumerate}

\subsection{Vorteile der Architektur}

Die gewählte Drei-Schichten-Architektur bietet für EventHub folgende Vorteile:
\begin{itemize}
  \item Sicherheit: Rechte- und DSGVO-Regeln werden zentral im Server
        durchgesetzt und können nicht clientseitig umgangen werden.
  \item Wartbarkeit: Benutzeroberfläche, Geschäftslogik und Datenhaltung
        können unabhängig voneinander weiterentwickelt werden.
  \item Skalierbarkeit: Die Server-Schicht kann horizontal skaliert werden,
        während die Datenbank separat optimiert werden kann.
  \item Portabilität: Web- und Mobile-Clients nutzen dieselbe REST-API.
\end{itemize}

\subsection{Komponentendiagramm}

\begin{figure}[H]
    \centering
    \includegraphics[width=0.5\textwidth]{Komponentendiagramm.png}
    \caption{Komponentendiagramm}
    \label{fig:Komponentendiagramm}
\end{figure}

Abbildung~\ref{fig:Komponentendiagramm} visualisiert die Drei-Schichten-Architektur von
EventHub als Komponentendiagramm. Die Client-Schicht (Web Client und Mobile App)
kommuniziert ausschließlich über HTTPS/JSON mit der REST-API der Server-Schicht.

In der Server-Schicht kapselt die \textit{Business Logic} die fachlichen Regeln und
Use-Case-Abläufe, während \textit{Authentification \& Authorization} die Authentifizierung und
rollenbasierte Zugriffsprüfung zentral durchsetzt. Persistente Daten werden in der
Datenbank-Schicht gespeichert; der Zugriff darauf erfolgt ausschließlich serverseitig,
wodurch Konsistenz- und Sicherheitsanforderungen zentral kontrolliert werden können.

\newpage

\subsection{REST-API-Spezifikation}

\subsubsection{API – Konventionen}

\begin{itemize}
  \item Base URL: \texttt{/api/v1}
  \item Datenformat: JSON (UTF-8)
  \item Authentifizierung: Token-basiert (Authorization: Bearer Token)
  \item Autorisierung: Rollen- und Rechteprüfung serverseitig
  \item Statuscodes: 200 OK, 201 Created, 204 No Content, 400 Bad Request,
        401 Unauthorized, 403 Forbidden, 404 Not Found, 409 Conflict
\end{itemize}

\subsubsection{Ressourcen und Endpunkte}

\paragraph{Authentifizierung und Benutzer}

\begin{longtable}{|p{5,4cm}|p{1,8cm}|p{7cm}|}
\hline
\textbf{Endpoint} & \textbf{Methode} & \textbf{Beschreibung} \\
\hline
\endfirsthead
\hline
\textbf{Endpoint} & \textbf{Methode} & \textbf{Beschreibung} \\
\hline
\endhead

/auth/register & POST & Benutzer registrieren (US-01) \\
\hline
/auth/login & POST & Benutzer anmelden (US-02) \\
\hline
/auth/2fa/verify & POST & Zwei-Faktor-Authentifizierung bestätigen \\
\hline
/auth/logout & POST & Benutzer abmelden \\
\hline
/users/me & GET & Eigenes Benutzerprofil abrufen (US-06) \\
\hline
/users/me & PATCH & Profil bearbeiten (US-04, US-07, US-08, US-09) \\
\hline
/users/me/privacy-consent & POST & DSGVO-Zustimmung speichern \\
\hline
\end{longtable}

\paragraph{Gruppen und Mitgliedschaften}

\begin{longtable}{|p{5,4cm}|p{1,8cm}|p{7cm}|}
\hline
\textbf{Endpoint} & \textbf{Methode} & \textbf{Beschreibung} \\
\hline
\endfirsthead
\hline
\textbf{Endpoint} & \textbf{Methode} & \textbf{Beschreibung} \\
\hline
\endhead

/groups & POST & Gruppe anlegen (US-10) \\
\hline
/groups & GET & Eigene Gruppen anzeigen (US-14) \\
\hline
/groups/{groupId} & GET & Gruppendetails anzeigen \\
\hline
/groups/{groupId} & DELETE & Gruppe löschen/archivieren (US-11) \\
\hline
/groups/{groupId}/join-code & POST & Beitrittscode verwalten (US-15) \\
\hline
/groups/{groupId}/members/join & POST & Gruppe per Code beitreten (US-12) \\
\hline
/groups/{groupId}/members/{userId} & PATCH & Mitglied sperren/Rolle ändern (US-15) \\
\hline
/groups/{groupId}/leave & POST & Gruppe verlassen (US-13) \\
\hline
\end{longtable}

\paragraph{Events}

\begin{longtable}{|p{5,4cm}|p{1,8cm}|p{7cm}|}
\hline
\textbf{Endpoint} & \textbf{Methode} & \textbf{Beschreibung} \\
\hline
\endfirsthead
\hline
\textbf{Endpoint} & \textbf{Methode} & \textbf{Beschreibung} \\
\hline
\endhead

/groups/{groupId}/events & POST & Event anlegen \\
\hline
/groups/{groupId}/events & GET & Events einer Gruppe anzeigen \\
\hline
/events/{eventId} & GET & Eventdetails anzeigen \\
\hline
/events/{eventId} & PATCH & Event bearbeiten (US-16) \\
\hline
/events/{eventId}/archive & POST & Event archivieren \\
\hline
/events/{eventId}/public & GET & Öffentliche Eventseite (US-39) \\
\hline
\end{longtable}

\paragraph{Notizen, Umfragen und Finanzen}

\begin{longtable}{|p{5,4cm}|p{1,8cm}|p{7cm}|}
\hline
\textbf{Endpoint} & \textbf{Methode} & \textbf{Beschreibung} \\
\hline
\endfirsthead
\hline
\textbf{Endpoint} & \textbf{Methode} & \textbf{Beschreibung} \\
\hline
\endhead

/groups/{groupId}/notes & POST & Notiz anlegen (US-17, US-21, US-27) \\
\hline
/groups/{groupId}/notes & GET & Notizen anzeigen (US-20) \\
\hline
/notes/{noteId} & PATCH & Notiz bearbeiten (US-18) \\
\hline
/notes/{noteId} & DELETE & Notiz löschen (US-19) \\
\hline
/groups/{groupId}/polls & POST & Umfrage anlegen (US-22) \\
\hline
/polls/{pollId}/vote & POST & Abstimmen (US-23) \\
\hline
/groups/{groupId}/payments & POST & Einzahlung erfassen (US-25) \\
\hline
/groups/{groupId}/finance/summary & GET & Kassenübersicht (US-28) \\
\hline
\end{longtable}

\noindent
\\
Die REST-API stellt alle Systemfunktionen als versionierte Ressourcen bereit.
Die Kommunikation erfolgt ausschließlich über HTTPS mit JSON als Austauschformat.
Geschäftslogik sowie Sicherheits- und Datenschutzregeln werden serverseitig durchgesetzt.


\section{Verteilungs-Sicht}

\subsection{UML-Verteilungsdiagramm}

\begin{figure}[H]
    \centering
    \includegraphics[width=1.0\textwidth]{Verteilungsdiagramm.png}
    \caption{Verteilungsdiagramm}
    \label{fig:deployment}
\end{figure}

Abbildung~\ref{fig:deployment} zeigt die physische Verteilung der Systemkomponenten
von EventHub zur Laufzeit. Web-Client (Browser) und Mobile App laufen auf
Client-Geräten und kommunizieren ausschließlich über HTTPS/REST mit dem
Backend-Server.

Der Backend-Server hostet die REST-API sowie zentrale Komponenten für
Geschäftslogik und Authentifizierung/Autorisierung. Persistente Daten werden auf
einem separaten Datenbank-Server gespeichert, der ausschließlich vom Backend
angesprochen wird.

Zusätzlich sind optionale externe Dienste angebunden, darunter ein
Benachrichtigungsdienst (E-Mail/SMS/Push), ein Kalenderdienst sowie ein
Objektspeicher für Medieninhalte. Diese werden über gesicherte HTTPS-Schnittstellen
vom Backend aus integriert.


\section{Sicherheit und Datenschutz}

\subsection{STRIDE-Sicherheitsanalyse}

Die Sicherheitsanalyse des Systems erfolgt anhand des STRIDE-Modells. Betrachtet wird
der Systemkontext bestehend aus Web-/Mobile-Client, REST-API (Backend) und Datenbank.
\\
\begin{longtable}{|p{2cm}|p{4cm}|p{4cm}|p{5cm}|}
\hline
\textbf{STRIDE} & \textbf{Risiko} & \textbf{Beispiel} & \textbf{Gegenmaßnahmen} \\
\hline
\endfirsthead
\hline
\textbf{STRIDE} & \textbf{Risiko} & \textbf{Beispiel} & \textbf{Gegenmaßnahmen} \\
\hline
\endhead

S (Spoofing) &
Identitätsvortäuschung &
Account-Übernahme durch gestohlene Zugangsdaten &
Passwort-Hashing (Argon2/bcrypt), HTTPS, 2FA, Rate-Limiting, Token mit begrenzter Laufzeit \\
\hline

T (Tampering) &
Manipulation von Daten &
Manipulierte Requests mit fremden IDs &
Serverseitige Validierung, Autorisierungsprüfungen, Prepared Statements, Ownership-Checks \\
\hline

R (Repudiation) &
Abstreitbarkeit von Aktionen &
Benutzer bestreitet Zahlung oder Sperre &
Audit-Logging (wer/was/wann), serverseitige Zeitstempel, Request-IDs \\
\hline

I (Information Disclosure) &
Informationsleck &
Unbefugter Zugriff auf Gruppen- oder Profildaten &
RBAC, Datenminimierung, sichere Fehlerausgaben, Verschlüsselung in Transit und at Rest \\
\hline

D (Denial of Service) &
Dienstverweigerung &
Brute-Force-Login oder API-Spam &
Rate-Limits, Timeouts, Paging, WAF/Reverse-Proxy \\
\hline

E (Elevation of Privilege) &
Rechteausweitung &
Mitglied wird unberechtigt Organisator &
Serverseitige Rollenprüfung, getrennte Admin-Endpunkte, gültige Rollen-Constraints \\
\hline
\end{longtable}

\subsection{Schutzmaßnahmen}

\subsubsection{Authentifizierung}
Die Authentifizierung erfolgt über ein tokenbasiertes Verfahren (Bearer Token).
Passwörter werden ausschließlich gehasht und gesalzen gespeichert. Optional kann eine
Zwei-Faktor-Authentifizierung (2FA) aktiviert werden.

\subsubsection{Autorisierung}
Die Autorisierung erfolgt serverseitig über rollenbasierte Zugriffskontrollen (RBAC).
Zusätzlich wird bei jedem Zugriff geprüft, ob der Benutzer zur angeforderten Ressource
(z.\,B. Gruppe oder Event) gehört.

\subsubsection{Eingabevalidierung und Datenintegrität}
Alle Eingaben werden serverseitig validiert. Datenbankzugriffe erfolgen ausschließlich
über vorbereitete Statements oder ORM-Mechanismen, um Manipulationen zu verhindern.

\subsubsection{Transport- und Datensicherheit}
Die gesamte Kommunikation erfolgt über TLS (HTTPS). Persistente Daten werden verschlüsselt
gespeichert oder auf verschlüsselten Datenträgern abgelegt. Backups werden ebenfalls
verschlüsselt gespeichert.

\subsubsection{Logging und Monitoring}
Sicherheitsrelevante Aktionen wie Login-Versuche, Rollenänderungen oder Zahlungen werden
auditierbar protokolliert. Monitoring-Mechanismen erkennen ungewöhnliche Zugriffsmuster.

\subsection{Datenschutz und DSGVO-Umsetzung}

\subsubsection{Zugriffsschutz (Privacy by Design)}
Der Zugriff auf personenbezogene Daten ist ausschließlich berechtigten Benutzern möglich.
Öffentlich zugängliche Inhalte zeigen nur explizit freigegebene Informationen.

\subsubsection{Datenminimierung}
Es werden nur Daten erhoben und gespeichert, die für den Betrieb der Anwendung erforderlich
sind. Sensible Informationen wie Passwörter werden niemals im Klartext gespeichert.

\subsubsection{Einwilligung und Transparenz}
Die Zustimmung zur Datenschutzerklärung wird mit Zeitstempel und Versionsnummer
(\texttt{privacy\_accepted\_at}, \texttt{privacy\_version}) gespeichert und ist nachvollziehbar.

\subsubsection{Löschkonzept und Aufbewahrung}
Benutzer können ihr Konto löschen lassen. Personenbezogene Daten werden dabei gelöscht oder
anonymisiert. Beim Austritt aus Gruppen werden zugehörige Einzahlungen gemäß fachlicher
Vorgabe entfernt oder logisch gelöscht (Soft-Delete).

\subsubsection{Betroffenenrechte}
Das System unterstützt die Rechte auf Auskunft, Berichtigung und Löschung personenbezogener
Daten. Ein Export der eigenen Daten kann in strukturierter Form bereitgestellt werden.


\section{Arbeitsteilung}
\subsection{Aufgabenteilung \textbf{Felix Franke}}

\subsubsection{Fachliche Grundlagen}
\begin{itemize}
    \item Analyse der Aufgabenstellung und fachlichen Anforderungen
    \item Festlegung eines aussagekräftigen Produktnamens
    \item Identifikation und Beschreibung aller Akteure und Rollen
    \item Erstellung des Glossars der Fachdomäne
\end{itemize}

\subsubsection{Funktionale Anforderungen}

\begin{itemize}
    \item Definition funktionaler Gruppen (z. B. Benutzerverwaltung, Gruppenmanagement, Eventplanung)
    \item Ausformulierung aller User Stories je funktionaler Gruppe unter Angabe von Rolle, Ziel, Nutzen, Akzeptanzkriterien und MoSCoW-Priorisierung
    \item Ergänzung fehlender funktionaler Anforderungen (Plausibilitätsprüfung)
\end{itemize}

\subsubsection{Nicht-funktionale Anforderungen}

\begin{itemize}
    \item Beschreibung der Qualitätsmerkmale nach ISO 25010 (Usability, Sicherheit, Performance,
Wartbarkeit etc.)
    \item Rechtliche Anforderungen (DSGVO)
\end{itemize}

\subsubsection{GUI \& Usability}

\begin{itemize}
    \item Entwurf der GUI‑Mockups für Webbrowser
    \item Entwurf der GUI‑Mockups für mobile Endgeräte
    \item Zustandsdiagramme für die Navigation
    \item Zuordnung der Screens zu den jeweiligen User Stories
    \item Beschreibung der angewandten UI-/UX‑Designprinzipien
\end{itemize} 

\subsection{Aufgabenteilung \textbf{Markus Wurms}}

\subsubsection{Domänen- \& Datenmodellierung}

\begin{itemize}
    \item Ableitung der Domänenklassen aus den User Stories
    \item Erstellung des fachlichen Domänenmodells (UML‑Klassendiagramm)
    \item Modellierung von Beziehungen, Multiplizitäten und Vererbungen
    \item Ergänzung sinnvoller Attribute und Klassen
\end{itemize}

\subsubsection{Zustände \& Datenpersistenz}

\begin{itemize}
    \item Zustandsdiagramme für zentrale Klassen (z. B. Event, Rechnung, Aufgabe)
    \item Überführung des fachlichen Modells in ein logisches Datenmodell
    \item Definition von Bewegungsdaten (Zahlungen, Buchungen, Statusänderungen)
    \item Erstellung der CRUD‑Matrix (User Stories $\leftrightarrow$ Klassen)
\end{itemize}

\subsubsection{Systemarchitektur \& Schnittstellen}

\begin{itemize}
    \item Beschreibung der Drei‑Schichten‑Architektur (Client – Server – Datenbank)
    \item UML‑Verteilungsdiagramm (Hardware-/Systemlandschaft)
    \item Definition der REST‑API (Ressourcen, Endpunkte, HTTP‑Methoden)
    \item Sequenzdiagramme für ausgewählte Kern-Use‑Cases
\end{itemize}

\subsubsection{Sicherheit \& Datenschutz}

\begin{itemize}
    \item Sicherheitsanalyse mittels STRIDE
    \item Beschreibung der Schutzmaßnahmen (Authentifizierung, Autorisierung, Verschlüsselung)
    \item Technische Umsetzung von DSGVO‑Anforderungen (Zugriffsschutz, Datenminimierung,
Löschkonzepte)
\end{itemize}



\section{Übersicht verwendeter Hilfsmittel}

\subsection{OpenAI ChatGPT (openai.com, chatgpt.com)}

\subsubsection{Nutzung von ChatGPT zur Formatierung}
ChatGPT wurde zur Hilfe gezogen beim Formatieren dieses Dokuments, insbesondere bei den Tabellen der User-Stories.
Außerdem haben wir mit ChatGPT im ganzen Dokument auf Rechtschreibfehler kontrolliert.

\subsubsection{Generierung von Mockups}
ChatGPT wurde zur Generierung von Mockups genutzt. Die jeweils verwendeten Prompts werden hier ausgelistet. Jedes generierte Mockup wurde per Bildbearbeitungsprogramm meist noch ein wenig verändert. Die Prompts selber wurden per Hand erstellt und in ein einfaches Eingabefeld in ChatGPT 5.2 eingefügt (Version Stand 05.02.2026).

\subsubsection{Prompts}

\paragraph{Login \& Registrierung}
\textbf{Prompt:}
\begin{prompt}
Erstelle ein wireframe-ui mockup einer desktop-website mit dem Namen "EventHub". Grauskala, keine Designelemente, nur einfache Rechtecke und Text-Label.

Layout:
desktop browser window {
zentrierter content-bereich ohne sidebar,

oben logo und schriftzug "EventHub",

darunter rechteckige login-box mit title "Login",

in der box mehrere rechteckige eingabefelder:

E-Mail

Passwort

darunter button "Login",

darunter text-links:

"Registrieren"

"Passwort vergessen",

 checkbox "Angemeldet bleiben",

 kleiner hinweisbereich "2-Faktor-Authentifizierung Code",

unter der login-box eine zweite kleinere box oder link "Anmelden"
}

Stil: flach, minimal, nur rechtecke und platzhaltertext, kein realistisches UI-design, kein schatten, keine farben, nur wireframe.
\end{prompt}

\paragraph{Dashboard}
\textbf{Prompt:}
\begin{prompt}
Erstelle ein wireframe-ui mockup einer desktop-website mit dem Namen "EventHub". Grauskala, keine Designelemente, nur einfache Rechtecke und Text-Label.

Layout:
desktop browser window {
top header bar { "EventHub" Schriftzug und Logo links und Navigation mit "Dashboard", "Gruppen", "Aufgaben", "Profil" rechts }
sidebar navigation on the left { "Dashboard" (active), "Gruppen", "Aufgaben", "Finanzen", "Profil" }
mittiger Content-Bereich rechts neben der Sidebar {
title: "Dashboard",

darunter mehrere übersichts-karten in einem grid-layout:

karte "Meine Gruppen" -> liste von 3 rechteckigen listen-objekten mit Gruppennamen

karte "Nächste Termine" -> liste von 3 rechteckigen listen-objekten mit Datum + Titel

karte "Offene Aufgaben" -> liste von 3 rechteckigen listen-objekten mit Aufgabenname + Status

karte "Benachrichtigungen" -> liste von 2-3 rechteckigen meldungen

oben rechts ein button "+ Gruppe erstellen"
}
}

Stil: flach, minimal, nur rechtecke und platzhaltertext, kein realistisches UI-design, kein schatten, keine farben, nur wireframe.
\end{prompt}

\paragraph{Gruppenübersicht}
\textbf{Prompt:}
\begin{prompt}
Erstelle ein wireframe-ui mockup einer desktop-website mit dem Namen "EventHub". Grauskala, keine Designelemente, nur einfache Rechtecke und Text-Label.

Layout:
desktop browser window {
top header bar { "EventHub" Schriftzug und Logo links und Navigation mit "Dashboard", "Gruppen", "Aufgaben", "Profil" rechts }
sidebar navigation on the left { "Gruppen" (active), "Dashboard", "Aufgaben", "Finanzen", "Profil" }
mittiger Content-Bereich rechts neben der Sidebar {
title: "Gruppenübersicht",

oben eine suchleiste "Gruppe suchen" und ein eingabefeld "Beitrittscode eingeben" mit button "Beitreten",

darunter sektion "Meine Gruppen" → liste von 4 rechteckigen listen-objekten mit:

Gruppenname

kurze Beschreibung

1x Button "Öffnen"

1x Button "Verlassen",

oben rechts ein button "+ Gruppe erstellen"
}
}

Stil: flach, minimal, nur rechtecke und platzhaltertext, kein realistisches UI-design, kein schatten, keine farben, nur wireframe.
\end{prompt}

\paragraph{Event-Detailseite}
\textbf{Prompt:}
\begin{prompt}
Erstelle ein wireframe-ui mockup einer desktop-website mit dem Namen "EventHub". Grauskala, keine Designelemente, nur einfache Rechtecke und Text-Label. Layout: desktop browser window { top header bar { "Eventhub" Schriftzug und logo links und navigation mit "Dashboard", Gruppen", "Aufgaben", "Profil" rechts} sidebar navigation on the left {"Übersicht "(active), "Notizen/Umfragen", "Aufgaben", "Finanzen", "Medien", "Mitglieder"}Mittiger Content-Bereich recht neben der sidebar {title: "Event-Detail", darunter eine Zeile mit "Event-Titel" und "Ort,Datum", darunter action buttons: "+Notiz erstellen", "+Aufgabe hinzufügen", "Mitglieder verwalten", darunter Sektion "Übersicht" -> untergeordnet liste von 3 rechteckigen listen-objekten mit "Notizen/Aufgaben" als Titel und "Beschreibung der Notiz/Aufgabe" als untertext"}"
\end{prompt}

\paragraph{Finanzen}
\textbf{Prompt:}
\begin{prompt}
Erstelle ein wireframe-ui mockup einer desktop-website mit dem Namen "EventHub". Grauskala, keine Designelemente, nur einfache Rechtecke und Text-Label.

Layout:
desktop browser window {
top header bar { "EventHub" Schriftzug und Logo links und Navigation mit "Dashboard", "Gruppen", "Aufgaben", "Profil" rechts }
sidebar navigation on the left { "Übersicht", "Notizen/Umfragen", "Aufgaben", "Finanzen" (active), "Medien", "Mitglieder" }
mittiger Content-Bereich rechts neben der Sidebar {
title: "Finanzen",

oben budget-übersicht als rechteckige info-boxen:

"Budget-Ziel"

"Aktueller Stand"

"Restbetrag"

darunter action buttons: "+ Einzahlung hinzufügen", "+ Ausgabe erfassen", "Rückzahlung auslösen",

darunter tabelle mit mehreren rechteckigen listen-objekten:

Datum

Typ (Einzahlung/Ausgabe)

Beschreibung

Betrag

Buttons "Bearbeiten", "Löschen"

optional einfache fortschrittsleiste als rechteckiger balken für Budgetfortschritt
}
}

Stil: flach, minimal, nur rechtecke und platzhaltertext, kein realistisches UI-design, kein schatten, keine farben, nur wireframe.
\end{prompt}

\paragraph{Notizen \& Umfragen}
\textbf{Prompt:}
\begin{prompt}
Erstelle ein wireframe-ui mockup einer desktop-website mit dem Namen "EventHub". Grauskala, keine Designelemente, nur einfache Rechtecke und Text-Label.

Layout:
desktop browser window {
top header bar { "EventHub" Schriftzug und Logo links und Navigation mit "Dashboard", "Gruppen", "Aufgaben", "Profil" rechts }
sidebar navigation on the left { "Übersicht", "Notizen/Umfragen" (active), "Aufgaben", "Finanzen", "Medien", "Mitglieder" }
mittiger Content-Bereich rechts neben der Sidebar {
title: "Notizen & Umfragen",

oben action buttons: "+ Notiz erstellen", "+ Umfrage erstellen",

darunter filterleiste mit dropdown "Typ filtern" und suchfeld,

darunter liste von 4 rechteckigen listen-objekten mit:

Titel der Notiz/Umfrage

kurzer Beschreibungstext

Status oder Anzahl Stimmen

Buttons "Bearbeiten", "Löschen", 

bei einer geöffneten Umfrage zusätzlich rechteckige Antwortoptionen mit Button "Ja" und Button "Nein" 
}
}

Stil: flach, minimal, nur rechtecke und platzhaltertext, kein realistisches UI-design, kein schatten, keine farben, nur wireframe.
\end{prompt}

\paragraph{Aufgabenplanung}
\textbf{Prompt:}
\begin{prompt}
Erstelle ein wireframe-ui mockup einer desktop-website mit dem Namen "EventHub". Grauskala, keine Designelemente, nur einfache Rechtecke und Text-Label.

Layout:
desktop browser window {
top header bar { "EventHub" Schriftzug und Logo links und Navigation mit "Dashboard", "Gruppen", "Aufgaben", "Profil" rechts }
sidebar navigation on the left { "Übersicht", "Notizen/Umfragen", "Aufgaben" (active), "Finanzen", "Medien", "Mitglieder" }
mittiger Content-Bereich rechts neben der Sidebar {
title: "Aufgabenmanagement",

oben action button: "+ Aufgabe erstellen",

darunter filterleiste mit dropdown "Status filtern" und suchfeld,

darunter aufgabenliste als mehrere rechteckige listen-objekte mit:

Aufgabenname

kurze Beschreibung

zugewiesene Person

Status-Anzeige (offen / in Bearbeitung / erledigt)

Buttons "Übernehmen", "Bearbeiten", "Erledigt",

optional gruppiert in sektionen "Offen", "In Bearbeitung", "Erledigt"
}
}

Stil: flach, minimal, nur rechtecke und platzhaltertext, kein realistisches UI-design, kein schatten, keine farben, nur wireframe.
\end{prompt}

\paragraph{Profil \& Einstellungen}
\textbf{Prompt:}
\begin{prompt}
Erstelle ein wireframe-ui mockup einer desktop-website mit dem Namen "EventHub". Grauskala, keine Designelemente, nur einfache Rechtecke und Text-Label.

Layout:
desktop browser window {
top header bar { "EventHub" Schriftzug und Logo links und Navigation mit "Dashboard", "Gruppen", "Aufgaben", "Profil" rechts }
sidebar navigation on the left { "Dashboard", "Gruppen", "Aufgaben", "Finanzen", "Profil" (active) }
mittiger Content-Bereich rechts neben der Sidebar {
title: "Profil & Einstellungen",

sektion "Profilinformationen" mit mehreren rechteckigen eingabefeldern:

Name

E-Mail

Passwort ändern

Speichern Button,

sektion "Benachrichtigungen" mit checkboxen oder toggles:

E-Mail

Push

SMS,

sektion "Sprache" mit dropdown auswahl,

sektion "Integrationen" mit zwei rechteckigen boxen:

"Kalender synchronisieren" mit Button "Verbinden"

"Sprachassistent koppeln" mit Button "Verbinden",

unten Button "Logout"
}
}

Stil: flach, minimal, nur rechtecke und platzhaltertext, kein realistisches UI-design, kein schatten, keine farben, nur wireframe.
\end{prompt}

\paragraph{Dienstleister-Dashboard}
\textbf{Prompt:}
\begin{prompt}
Erstelle ein wireframe-ui mockup einer desktop-website mit dem Namen "EventHub". Grauskala, keine Designelemente, nur einfache Rechtecke und Text-Label.

Layout:
desktop browser window {
top header bar { "EventHub" Schriftzug und Logo links und Navigation mit "Dashboard", "Anfragen", "Angebote", "Rechnungen", "Ressourcen", "Profil" rechts }
sidebar navigation on the left { "Dashboard" (active), "Anfragen", "Angebote", "Rechnungen", "Material", "Mitarbeiter", "Profil" }
mittiger Content-Bereich rechts neben der Sidebar {
title: "Dienstleister Dashboard",

darunter mehrere übersichts-karten in einem grid-layout:

karte "Neue Anfragen" → liste von 3 rechteckigen einträgen mit Eventname + Datum + Button "Öffnen"

karte "Offene Angebote" → liste von 2 rechteckigen einträgen mit Preis + Status

karte "Offene Rechnungen" → liste von 2 rechteckigen einträgen mit Betrag + Fälligkeitsdatum

karte "Ressourcenstatus" → einfache liste von Material oder Mitarbeitern mit Verfügbarkeit

}
}

Stil: flach, minimal, nur rechtecke und platzhaltertext, kein realistisches UI-design, kein schatten, keine farben, nur wireframe.
\end{prompt}

\paragraph{Dashboard – Mobil}
\textbf{Prompt:}
\begin{prompt}
Erstelle ein wireframe-ui mockup einer mobilen app-version der website mit dem Namen "EventHub". Hochformat, smartphone-format, Grauskala, keine Designelemente, nur einfache Rechtecke und Text-Label.

Layout:
smartphone screen im hochformat {

oben eine schmale header bar mit schriftzug "EventHub",

mittiger content-bereich mit vertikal gestapelten karten:

karte "Meine Gruppen" → liste von 2–3 rechteckigen einträgen

karte "Nächste Termine" → liste von 2 rechteckigen einträgen

karte "Offene Aufgaben" → liste von 2 rechteckigen einträgen

karte "Benachrichtigungen" → liste von 2 rechteckigen meldungen

unten feste bottom-navigation bar mit 4 icons/text-labels:
"Dashboard" (active), "Gruppen", "Aufgaben", "Profil"

alle elemente groß und touch-geeignet, einfache rechtecke, viel abstand zwischen elementen
}

Stil: flach, minimal, nur rechtecke und platzhaltertext, kein realistisches UI-design, kein schatten, keine farben, nur wireframe.
\end{prompt}

\paragraph{Event-Detail – Mobil}
\textbf{Prompt:}
\begin{prompt}
Erstelle ein wireframe-ui mockup einer mobilen app-version der website mit dem Namen "EventHub". Hochformat, smartphone-format, Grauskala, keine Designelemente, nur einfache Rechtecke und Text-Label.

Layout:
smartphone screen im hochformat {

oben header bar mit titel "Event-Detail" und zurück-button,

darunter event-informationen als block:

Event-Titel

Datum

Ort

darunter horizontale tab-leiste oder segmentierte navigation mit:
"Übersicht" (active), "Notizen", "Aufgaben", "Finanzen",

darunter scrollbarer content-bereich mit mehreren rechteckigen listen-objekten (Notizen/Aufgaben),

unten zwei große touch-buttons:
"+ Notiz"
"+ Aufgabe"

alle elemente groß, vertikal gestapelt, fingerfreundliche abstände, einfache rechtecke, kein desktop-layout
}

Stil: flach, minimal, nur einfache rechtecke und platzhaltertext, kein realistisches UI-design, kein schatten, keine farben, nur wireframe.
\end{prompt}

\paragraph{Aufgaben – Mobil}
\textbf{Prompt:}
\begin{prompt}
Erstelle ein wireframe-ui mockup einer mobilen app-version der website mit dem Namen "EventHub". Hochformat, smartphone-format, Grauskala, keine Designelemente, nur einfache Rechtecke und Text-Label.

Layout:
smartphone screen im hochformat {

oben header bar mit titel "Aufgaben",

darunter filterleiste mit dropdown "Status" und suchfeld,

darunter vertikal gestapelte aufgabenliste mit mehreren großen rechteckigen einträgen:

Aufgabenname

kurze Beschreibung

zugewiesene Person

Statusanzeige

rechts oder unten pro eintrag große touch-buttons:
"Übernehmen"
"Erledigt",

unten feste bottom-navigation bar mit:
"Dashboard", "Gruppen", "Aufgaben" (active), "Profil"

alle elemente groß, fingerfreundlich, viel abstand, nur einfache rechtecke
}

Stil: flach, minimal, nur rechtecke und platzhaltertext, kein realistisches UI-design, kein schatten, keine farben, nur wireframe.
\end{prompt}

\paragraph{Finanzen – Mobil}
\textbf{Prompt:}
\begin{prompt}
Erstelle ein wireframe-ui mockup einer mobilen app-version der website mit dem Namen "EventHub". Hochformat, smartphone-format, Grauskala, keine Designelemente, nur einfache Rechtecke und Text-Label.

Layout:
smartphone screen im hochformat {

oben header bar mit titel "Finanzen",

darunter budget-übersicht als große info-boxen:

"Budget-Ziel"

"Aktueller Stand"

"Restbetrag"

darunter zwei große touch-buttons nebeneinander oder gestapelt:
"+ Einzahlung"
"+ Ausgabe",

darunter vertikal gestapelte liste von einträgen mit:

Datum

Beschreibung

Betrag (+ oder -)

kleiner button "Details",

unten feste bottom-navigation bar mit:
"Dashboard", "Gruppen", "Aufgaben", "Profil"

alle elemente groß, vertikal, fingerfreundlich, nur einfache rechtecke und text
}

Stil: flach, minimal, nur rechtecke und platzhaltertext, kein realistisches UI-design, kein schatten, keine farben, nur wireframe.
\end{prompt}

\paragraph{Rollen- und Rechteverteilung}
\textbf{Prompt:}
\begin{prompt}
Erstelle ein wireframe-ui mockup einer desktop-website mit dem Namen “EventHub”. Grauskala, keine Designelemente, nur einfache Rechtecke und Text-Label.

Layout:
desktop browser window {
top header bar { “EventHub” Schriftzug links, Navigation “Dashboard”, “Events”, “Benutzer”, “Admin” }
sidebar navigation on the left { “Übersicht”, “Events”, “Benutzer”, “Rollen & Rechte” (active) }

mittiger Content-Bereich rechts neben der Sidebar {
title: “Rollen- und Rechteverwaltung”,

oben dropdown “Institution auswählen”,

darunter tabelle mit mehreren rechteckigen zeilen:
Spalten:
Benutzer | Rolle | Lesen | Schreiben

Spalte Benutzer enthält Benutzernamen,
Spalte Rolle enthäht Platzhalter Rollenname,
Spalten lesen und schreiben sind checkboxen,


oben rechts button “+ Rolle vergeben”,
unten rechts button “Speichern”
}
}

Stil: flach, minimal, nur rechtecke und platzhaltertext, kein realistisches UI-design, kein schatten, keine farben, nur wireframe.
\end{prompt}

\paragraph{Eventanfrage}
\textbf{Prompt:}
\begin{prompt}
Erstelle ein wireframe-ui mockup einer desktop-website mit dem Namen "EventHub". Grauskala, keine Designelemente, nur einfache Rechtecke und Text-Label.

Layout:
desktop browser window {
zentrierter content-bereich ohne sidebar {

title: "Eventanfrage stellen",

mehrere rechteckige formularfelder:

Firmenname / Name

Kontakt E-Mail

Eventdatum

Ort

Budget

Beschreibung / Anforderungen (großes Textfeld)

darunter button "Anfrage senden"
}
}

Stil: flach, minimal, nur rechtecke und platzhaltertext, kein realistisches UI-design, kein schatten, keine farben, nur wireframe.
\end{prompt}

\paragraph{Eventübersicht (Lesemodus)}
\textbf{Prompt:}
\begin{prompt}
Erstelle ein wireframe-ui mockup einer desktop-website mit dem Namen "EventHub". Grauskala, keine Designelemente, nur einfache Rechtecke und Text-Label.

Layout:
desktop browser window {
top header bar { "EventHub" }

sidebar navigation on the left { "Übersicht" (active), "Notizen", "Aufgaben", "Finanzen" }

mittiger Content-Bereich {
title: "Eventübersicht (Lesemodus)",

liste von einträgen ohne buttons oder bearbeitungsoptionen,

kleiner hinweistext oben: "Schreibrechte deaktiviert – nur Lesen möglich"
}
}

Stil: flach, minimal, nur rechtecke und platzhaltertext, kein realistisches UI-design, kein schatten, keine farben, nur wireframe.
\end{prompt}




\end{document}